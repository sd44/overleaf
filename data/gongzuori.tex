\chapter{草稿之一——工作制}

\section{传统超时加班企业}

在中国,劳动密集型企业,尤其是制造业,是我们传统的超时加班重灾
区\cite{guojilaogongshijian},曾经引发社会极大关注的富士康连续跳楼事件便被认为是
因超时加班所致,仅2010年一年被媒体曝光出来的就有“14连跳”。富士康为降低、杜
绝此类事件采取了一系列举措,2010年、2011年两次大幅提升工资、健全加班制,铺设
大面积防跳楼网,签订《不自杀协议》等。2011年后,就很少有相关报道了……

根据《富士康工资、工时与生产管理调研》\cite{fushikangzuixin}一文,虽然至2015年为
止,富士康工人超时加班现象仍较严重,但大量裁员现象却与之并存。“2013年媒体又
报道了富士康新一轮的变相裁员浪潮(在2013年富士康全国用工规模减少了21万),引
起工人以`跳楼'”或停工等形式的激烈反抗。”。2016年,BBC发文报道《富士康用机器
人取代了6万名工人》。

\section{新型超时加班企业}

金融和互联网企业是二十一世纪新加班重灾区。华为在舶来词方面有个贡献,将日
本“过劳死”的概念成功传播到了中国,“加班文化”流传甚广,有很多段子。华为
苏州研究院椅子背后常备的睡袋,酷爱加班文化的日本专家入职华为两个月后愤而辞
职“你们这样是不人道的”,华为总裁任正非挽留要回北京陪妻子的副总李玉琢时
说“这样的老婆你要她干什么?”等。

说起创意,华为也一点也不比富士康《不自杀协议》差。我国于2007年6月29日颁布《劳
动合同法》,其中第十四条规定“有下列情形之一,劳动者提出或者同意续订、订立劳
动合同的,除劳动者提出订立固定期限劳动合同外,应当订立无固定期限劳动合同:
(一)劳动者在该用人单位连续工作满十年的……”。这个法案将于2008年1月1日正式
实施,在2007年底,华为安排7000余名工龄8年以上的老工人向公司递交自请辞信,作为
补偿,华为向这些老工人支付约十亿元违约金,然后再重新聘用这些“失业工人”,工
龄从零开始重新计算。数天后,2008年1月1日,工人们“自愿”为了几万到十几万的补
偿金,放弃了订立“无固定期限合同”的可能,变为了1至3年劳动合同的新工人。“华
为裁员门事件后,沃尔玛、环球、摩托罗拉等公司也先后进行了应对新法的人力资源调
整。《劳动合同法》实施前的“阵痛”似不减反增。\cite{huaweimaiduan}

2010年8月,华为“公司14级以上工人被要求`自愿'签署《成为奋斗者申请
书》……申请“华为奋斗者”有一个必备条件,需要添加“我申请成为与公司共同奋斗
的目标责任制工人,自愿放弃带薪年休假、非指令性加班费和\textbf{陪产假}”这句
话。\cite{huaweifendou}华为真是个互联网企业的好模板。

任正非在2001年有篇文章《华为的冬天》非常有名,当时国内大IT公司似乎基本没有这
种“Winter is coming”的论调,更无一认为自己可能马上狗带,他的危机意识立刻被
广大媒体、企业称赞。近20年过去,在这种强烈的危机意识下,华为仍是高奏凯歌、一
路前行。而2017年,一些年过34岁的交付工程维护人员,过40岁的研发员工和45岁的老
员工却可能迎来了真正的凛冬,要被清退掉去他处过日子了。当然,华为已经惯例辟谣
了。据说当前白色家电企业也已经开始学习华为,对34岁工程师劝退……

2016年8月29日起,58公司总裁兼CEO姚劲波的微博陆续被众多58公司工人、工人家属和
社会人士浏览并情绪不稳定地评论\cite{tai58},起因是58公司在不发邮件和公文的情况下
口头传播了公司新工作制度——“996工作制”,早九点上班,晚九点下班,星期六正常
上班,没有任何补贴。虽然这种工作制可能并非58公司原创,但却是由因它开始引起公
众反加班的社会影响,并且在事件后,“996工作制”并没有受到实际影响,反而成为了
不少单位一种明目张胆的制度。笔者认为,58公司“996工作制”这一事件,可以定为中
国劳动制度的一个里程碑。这标志着严重超时工作制从原来只是个别公司内部隐性文化、
不成文规定,发展至显性公司制度,并最终成为一种公开的可以被任何企业复制的社会
劳动制度。

滴滴出行于2016年底连发三篇大数据报告《2016年度加班最“狠”公司排行
榜》\cite{zuihen},涉及金融、互联网、公关、广告四个行业,包括33家公司。这33家公
司中最晚下班时间均在20点后,其中20点到21点之间下班的只有15家。四个行业相比较,
工作日时长方面金融行业较好,互联网业最差,最高加班时间是京东23点16分。周末上
班方面,金融业最差。0-5点下班返工人数方面,公关、广告公司表现突出,其中奥美广
告返工人数3080人,金宝大厦3家公司合计2102人。

\section{工作时长立法}

去看二十世纪或者当前的劳动法已经是难堪之事,那让我们粗略看下最早期的工时立法吧。

英国全行业立法是始于“1874年,R. A. Cross提出工厂方案,最终使得所有的英国工人
都享受10小时工作的权利”\pagescite[][96]{britishworkday}。

法国一步到位,直接是全行业立法,“法国1850年9月5日的十二小时工作日法令是临时
政府1848年3月2日法令的资产阶级化的翻版;这个法令适用于一切作
坊”\pagescite[][319]{capital}。

我们这些大体量的企业,靠着不懈的努力,向前150多年终于赶超到了英法19世纪中后期
水平,真是可歌可泣。另外,24小时工作制其实也已经来了,就是工作、睡觉、起床接
着干活,某为分公司椅子背后就挂着睡袋,随时一天24小时不离公司,祝这些企业能早
日赶超到19世纪早期劳动水平吧,就是不知道是否还需要童工呢?

结合实际来看,我国现在实行的1995年劳动法,规定的8小时工作制相比其他各国标准较
高,要求劳动时长较短,比德国、新加坡等一周60小时工作制还要少不少。在实际操作
上缺乏一些空间,可能的解决方案如弹性工作制、休息权等问题还在探讨中。希望尽快
出台相关政策。

\section{超时加班的危害和原理}
\label{sec:gzryuanli}

150多年前的《资本论》中提到的一些情况,居然仍高度适用我们的当前社会,在世界发
展方面来说这真不是一件庆幸之事。那么为何我们又重新回到150年前的工作状况?

马克思原文如下(不想看马克思原文的读者,可以略过这部分):
\begin{quotation}
  至于个人受教育的时间,发展智力的时间,履行社会职能的时间,进行社交活动的时
  间,自由运用体力和智力的时间,以至于星期日的休息时间,——这全都是废话!但
  是,资本由于无限度地盲目追逐剩余劳动,像狼一般地贪求剩余劳动,不仅突破了工
  作日的道德极限,而且突破了工作日的纯粹身体的极
  限。\pagescite[][306]{capital}

  《1861年爱尔兰面包业委员会的报告》中提到,“委员会认为,把工作日延长到12小
  时以上,是横暴地侵犯工人的家庭生活和私人生活,这就侵犯一个男人的家庭,使他
  不能履行他作为一个儿子、兄弟、丈夫和父亲所应尽的家庭义务,以致造成道德上的
  非常不幸的后果。12小时以上的劳动会损害工人的健康,使他们早衰早死,因而造成
  工人家庭的不幸,恰好在最必要的时候,失去家长的照料和扶
  持。”\pagescite[][292]{capital}

  工人阶级中就业部分的过度劳动,扩大了它的后备军\footnote{想在本行业入职的失业或半失
    业人}的队伍,而后者通过竞争加在就业工人身上的增大的压力,又反过来迫使就业
  工人不得不从事过度劳动和听从资本的摆布。工人阶级的一部分从事过度劳动迫使它
  的另一部分无事可做(无事可做指后备军),反过来,它的一部分无事可做迫使他的
  另一部分从事过度劳动,这成了各个资本家致富的手段,同时又按照与社会积累的增
  进相适应的规模加速了产业后备军的生产。\pagescite[][733]{capital}

  决定工资的一般变动的,不是工人人口绝对数量的变动,而是工人阶级分为\textbf{现役军
    和后备军的比例}的变动,是过剩人口相对量的增减,是过剩人口时而被吸收、时而
  又被游离的程度。\pagescite[][733]{capital}

  产业后备军在停滞和中等繁荣时期加压力于现役劳动军,在生产过剩和亢进时期又抑
  制现役劳动军的要求。所以,\textbf{相对过剩人口}是劳动供求规律借以运动的背景。它把这
  个规律的作用范围限制在绝对符合资本的剥削欲和统治欲的界限之
  内。”\pagescite[][736]{capital}

  不变资本的固定部分即工厂建筑物、机器等等的规糕,不管用来工作16小时,还
  是12小时,都会仍旧不变。工作日的延长并不要求在不变资本的这个最花钱的部分上
  有新的支出。此外,固定资本的价值,由此会在一个较短的周转期间系列中再生产出
  来,因而,这种资本为获得一定利润所必须预付的时间缩短了。因此,甚至在额外时
  间支付报酬,而且在一定限度内甚至比正常劳动时间支付较高报酬的情况下,工作日
  的延长都会提高利润。因此,现代工业制度下不断增长的增加固定资本的必要性,也
  就成了唯利是图的资本家延长工作日的一个主要动力。\pagescite[][91]{capital3}

  资本主义生产方式按照它的矛盾的、对立的性质,还把浪费工人的生命和健康,压低
  工人的生存条件本身,看做不变资本使用上的节约,从而看做提高利润率的手
  段。\pagescite[][101]{capital3}
\end{quotation}

以上意思是指:
\begin{enumerate}
\item 工人、职员异化成为企业身上一个微小的、易损坏和易更换替代的一个器官。不止个
  人的健康、生命受到损害,他的社交角色、儿女角色、父母角色、夫妻角色等社会角
  色则被严重削弱和抑制。由此会引发什么问题呢?在中国传统的家庭人伦关系上,难
  以孝敬、赡养老人,难以维护保养夫妻关系,难以同子女产生良好沟通和教育,家庭
  组建、生儿育女的时间也都会滞后许多。尤其是华为《奋斗者申请书》居然明目张
  胆“自愿”要求放弃本就不多的陪产假,真是反基本社会人伦,千夫可指。当然,华
  为始终横眉冷对,淡定得很,人家负面新闻也像富士康一样越来越少了,这真是进步
  呵!

\item 在职工人、职员的过度劳动,使当前就业人数相对于正常劳动情况下应当就业的人数
  减少了。当前未就业或者半就业工人、职员的存在又使在职工人、员工的工资报酬被
  压低。

\item 随着生产力发展,机器等固定资本的支出愈加巨大,不管其使用或不适用,都在产生
  折耗,这是生产力发展所必要的。为了事实上节约这种固定资本,就需要延长工人劳
  动时间,这就使工人超长劳动,乃至24小时劳动似乎成为一种生产力发展的必须。另
  外,如果雇佣更多工人,则需要更多数量和更大投资的固定资本,这对本已庞大的不
  变资本支出无异于雪上加霜。

\item 此外,脱离马克思文本,就实际情况来看,通过超长工作日所产生的工人、职员数量
  节约,使企业培训成本、培养成本、管理成本、福利保障成本、风险成本均大幅降
  低。
\end{enumerate}

综上,企业倾向于让员工尽量加班而非在原工作时长不变的情况下雇佣更多工人。

% 那么工人、职员为什么要加班呢?\begin{enumerate}

% \item 如富士康这类工厂,将基本工资设的较低,工人的工资财富积累常常只能在加班费中
%   来完成。不加班,不赚钱。加班了,才有钱。这是两百来年很多大工厂惯用不变的伎
%   俩。或者如华为将部分加班费融合进了工资,一些职员由此认为工资是可观的。这两
%   种情况其实都是一样的,就是企业在员工应得的工资中抽取了一部分给工人。

% \item 如华为这类企业,已在社会具有相当的名声。大众普遍认为能进华为的,肯定是有一
%   定水平的。能在华为干住的,跑到其他企业肯定是不怕累的。在一些职员看来,被压
%   榨就压榨吧,加班就加班吧,一个跳板而已。撑过现在,未来是光明的。

% \item 企业考核。不管是领导思想上的还是落在纸上的考核,都将加班量作为职员是否合格
%   的一个标准,华为有个词很有意思,叫“工作量饱满”。工作量不饱满,你就不是个
%   称职的员工。工作量饱满了,员工的个人时间,家庭时间也就别想饱满了。

% \end{enumerate}

员工作为这个微小的、易损坏的、易更换替代的工厂、企业小器官,在工作10余年被榨
干后,甚至只是在自己年龄增大、工龄福利增多后,就迎来自己被扫地出门的结局,他
们将被年龄更小,福利待遇更低的年轻员工所取代!超长工时所带来的利润诱惑是极易
在工厂、企业间传染的。这便是自由!但这是资本的自由,而非人的自由!

\section{工时改革问题的民族国家与全球化困局}

重新结合实际情况立法,对工人、工会赋权,加入弹性工作制和阶梯性休假等补偿措施,
加大执法力度,主动执法,对较重违规现象进行惩罚性罚款等,这些举措理论上将能有
效解决工时严重过长的问题。

就以往传统历史经验来说,19世纪英国工作日改革经验表明,“劳动生产率和劳动强度
的变化,或者是在工作日缩短以前,或者是紧接着在工作日缩短以后发生
的”\pagescite[][601]{capital},后来福特8小时工作制以及世界上广泛传播的劳动法
均在一定程度上证明了这点。

但现实是严酷的,我们可能无法出台实质的修正措施。中国当今所面临的正是资本语境
中,民族国家与全球化这一巨大张力困境。

所谓全球化,说穿了就是资本在全球高速通畅的流动。不管是中国政府,还是美国政府,
不管是工人还是大企业家在面对全球化这一庞然大物时都常是有心乏力。东南亚、南亚、
拉美劳动市场的低廉——相当程度上是因为当地贫困人口、童工、更贫穷国家的劳动力
输入;政府的非现代化管控要求,法律法规不健全;

所谓民族国家,则是强调本国利益,打击抵制他国相对本国的优势力量。2008年经济危
机后,世界各国看到了中国制造崛起的强势,打造各种非自由的贸易壁垒,对中国产品
出口设置各种障碍,征收各种不合理高额关税,发起各种非法反倾销反垄断诉讼、调查,
利诱逼迫并用迫使中国大型企业外迁;种种境况下,中国资本的外逃量恐不是少数。


以富士康\cite{foxconnwiki}为例,其在巴西、匈牙利、斯洛伐克、土耳其、捷克、日本、
马来西亚、墨西哥、印度、美国均有工厂。印度方面已与富士康签订谅解备忘录,富士
康预计在5年之内投资印度50亿美元,“美国方面唐纳德·特朗普总统于2017年7月26日
宣布,富士康将在威斯康星州东南部建立一个价值100亿美元的平板电视制造工厂。威斯
康星州将每年向富士康支付高达2.5亿美元的补贴,为期十五年。这笔交易被一些人批评
为是拿取30亿美元纳税人税收资助激励富士康。威斯康星州立法机关的无党派预算办公
室的分析确定,国家纳税人将在2043年收回投资。”

外国对中国的强力抵制,并不是因为中国的意识形态与他们不同,不是中国“邪恶”,
只因我们可能变得越加强大!

这真是个困局呵。安东尼·吉登斯在《现代性的后果》一书中提出,现代性的全球扩展
趋向产生了一个“失控的世界“,它的出现没有人也没有政府能够全面地控制。马克思
用怪物来描述现代性,而吉登斯将其比作坐在巨型汽车或”猛兽“上面……人类的聪明
才智,在这里居然毫无用处,是我们人类自己,孕产出了种种怪胎……

当马克思、恩格斯研究人类学时,当吉登斯希望发现有着未知社会模式的原始部落时,
都是这样一种绝望下的希望。虽然希望,却也绝望;虽然绝望,却也希望……

最后,用工人诗人许立志的一首诗来结束本节吧。

\poemtitle*{流水线上的兵马俑} \settowidth{\versewidth}{Than Tycho Brahe, or Erra Pater:}
\begin{verse}[\versewidth] \kaishu
  沿线站着\\
  \qquad 夏丘\ \qquad \quad 张子凤\qquad 肖朋\\ \qquad 李孝定 \qquad 唐秀猛\qquad 雷兰娇\\ \qquad 许立志 \qquad 朱正武\qquad \\ \qquad 潘霞\ \qquad \quad 苒雪梅\\这些不分昼夜的打工者 \qquad 穿戴好\\ \qquad 静电衣 \qquad 静电帽 \quad 静电鞋\\ \qquad 静电手套 \quad 静电环 \\整装待发 \quad 静候军令\\只一响铃功夫\\悉数回到秦朝
\end{verse}
\newcommand{\attrib}[1]{\nopagebreak{\raggedleft\small #1\par}} \attrib{许立志 (1990--2014)\qquad}


%%% Local Variables:
%%% mode: latex
%%% TeX-master: "../main"
%%% End:
