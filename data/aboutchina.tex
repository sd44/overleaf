\chapter{前言}

\section{中国不能乱}

(丁家庄的空间生产和中国的空间生产系列暂时搁置,因笔者深感有必要先从中国历史入
手,合适时机这三部分将穿插进行。)

中国近几十年的发展有目共睹,我们的衣食住行得到了飞跃式的提高,坚硬的城乡二元壁垒
被打破,阶级流动前所未有,物质生活日益丰富,诸如此类。但同时我们也面临民族国家与
全球化范畴里日益巨大的挑战,人民不容易,国家也确实不容易。

资本主义的危机不能被消灭,只能内内外外转移来转移去。这一准则在民族国家与全球资本
化语境里的应用使世界政治的竞技场愈加残酷。特别是在2008年世界经济危机之后,虽然中
国因强有力的国家管制未受它国那样损失,但它国已从全球化一端向民族国家一端加速倾斜,
使中国的世界市场地位受到更大挑战。特朗普上台后,更让美国倾向于民族国家立场,并利
用来盘剥全球,尤其是中国。形势压力下,民和国处境日艰,催生和加强一些激进思想。左
派中激进托派工人运动思想和右派资本自由化思想均是如此,而金融集团欲让中国彻底新自
由主义化、全球化以有利于本集团利益。
<!--more-->

国外反动势力意图阴谋颠覆中国之心未曾停歇。在中国发生重大不良社会影响事件时,原本
极右反华的某些国外网站、媒体一下子热情迎接了我国的左派。虽然国内一些不良事件中有
官僚用此作为借口试图避免事态扩大,但是整体事实却不曾改变,另外在这些事件中国外反
动势力即使不是主因,往往也是部分存在。

国人常将国家作为坚不可摧,或可凭高层领导的英明步步走高,这是不科学的。国家面临的
是一个错综复杂实践中的现实社会,有些选择必须要面对,但这些选择的选项在一些时候可
能都是灰色的,无一较为完善,如薄一波在《若干重大决策与事件的回顾(上下)》形容统
购统销时所说``两种`炸药'中的选择''。\pagescite[][259]{boyibo}

中国不能再激荡,中国不能乱!有问题和分歧要探讨和争辩,可以声音大些但要理智,尤其
要综合全局。人之艰难、国之不易,上层下层均要互相考量和体谅。拥护党和国家的领导,
采用合适渠道、方法是前提,过度激进要不得。

以阿根廷为例,当它面临秃鹫基金的绞杀时数次突围不得解,最终在以美国为首的部分发达
工业国家和世界金融集团压力下被迫偿债时,华尔街为阿根廷掉滴上一滴``善良的''眼泪后
立即弹冠相庆,华尔街鼓吹``从表面上看,似乎秃鹫基金是最大赢家。但实际上,这起官司
今天走到终点,对于阿根廷来说是个双赢的结果——\textbf{该国将最终得以重返国际资本市
  场},重新开展国际融资,以帮助经济走出困境。''\footnote{可参
  考\url{https://wallstreetcn.com/articles/230968},加粗部分并非此文作者一家之言,
  而是一些人的共同鼓吹。}多么好的双赢啊!阿根廷发行百年8\%的国债,阿根廷比索近乎
崩溃,阿根廷不再哭泣,它已经没有眼泪……我们国家的一些金融从业者对华尔街的一系列
操作热血沸腾,倾倒崇拜……

中国体量太大,营养太丰富,在当今残酷的世界政治竞技场中,一个伤痛的中国将被诸多国
家群拥而上,饕餮而食,渣骨不剩,万劫不复……不了解民族国家与全球资本化下世界政治
的这一残酷,要么是政治上幼稚无知,要么是包藏祸心,望大家明鉴,莫要让亲者痛、仇者
快。我们艰难地寻求出路吧。

\section{本篇立意}


人类世界一直缺乏对社会现实正视和直面的勇气,常有一层面纱蒙在真实和悲苦之上,或遮
掩或美化或扭曲……为应对悲苦人生,尼采所说日神阿波罗的幻梦便长期占据主要地位,现
代性的日益强大使之愈演愈烈。中国也是如此。只有敢于直面悲苦,尊重生命,尊重人之所
以为人,在纵情忘我的对生命原始冲动、创造的追求中才能去求得真实满足和慰
藉。\footnote{这其中也暗含一种悲观的可能——人是否能大量摆脱动物性和原始性……}

我们的历史和现实是那么复杂难懂吗?当代理论太渺小不能大量说明我们的历史和现实吗?
不是的!即使排除形态倾向过重、学术流氓和个人主观或迷幻寄托的著作,仍有许多著作在
反复说明一系列问题。但这些有水平的著作因其著作者深处社会环境利益相关中难以简明直
言。

“关于中国”这部分着重于通过理清中国历史脉络以及笔者对当今中国问题的个人不成熟见
解来激发人的思考,从而减弱幻梦、尊重真实、勇敢面对、正视错误、科学发展、强盛国家、
保障人民以及为了人类的美好明天。这一目的并非有笔者亲自实现,实际上这系列文章应该
是几乎无人问津外加毫无影响力,但笔者可以成为立志于从事这方面的一系列推动者中的一
员。

\section{本篇文章结构}

本系列文章历史方面多为史料来源的总结和引用,个别方面加以真诚直接地辩证批判。因这
个电子书项目是开源免费的个人项目,并无商业目的,应无侵权之忧,另外笔者会注明参考
文献。之所以大量引用,出自于简明整理与普通读者难以去搜集整理这些资料的考虑。

第x章到第x章为总结当代中国从1949建国年至今的经济重大事件,第x章为总结说明和笔者对
当今中国的思考建议。本系列着重参考了薄一波《若干重大决策与事件的回顾(上下)》,
邓书杰,李梅,吴晓莉和苏继红所著《中国历史大事详解丛书(套装共9册
)》\cite{zhengshujie}\footnote{本套书为电子书,无法注明具体页码},blablablabla以
及对这些参考资料的追根溯源。 


%%% Local Variables:
%%% mode: latex
%%% TeX-master: "../main"
%%% End:
