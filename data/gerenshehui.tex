\chapter{个人社会学}
\label{chap:gerenshehuixue}

对当前的时代,法兰克福学派和利奥塔曾提出过“资本主义可以吞噬一切”,对于资本现代性的批判更是多种多样,但是当前的社科理论和现实实践均未产生(事实上当前也不可能有)一个宏观、明确、完整和连贯的指向。全球、国家、组织,政治、经济、传媒、日常生活等均笼罩在被资本现代性规训的迷雾中。专家学者往往也困在这迷雾中,或求得私利,或止步不前,或提出希望但不能明确指向。

现代性所提供的庞杂信息以及强力规诫催生和加强了“规诫”的反面——个人\textbf{求“真”}的强烈诉求,而底层中,或者有过底层经历的个人,不管这种底层是物质的还是精神的,更可能产生这种诉求。这到底是一个什么样的世界?这个世界中诸多迷雾背后的真相到底是什么?诸多矛盾缘何产生又是怎样充满张力的融合起来?我缘何如此?应当如何改进?

笔者认为作为社会原子个人,虽受约束、限制和矛盾最多,但在一些方面相比组织、圈子,具有更强摆脱规诫的能力和愿望,也更加无畏,拥有极大地活跃性和生命力。如果诸多个人选择\textbf{个人社会学}道路,在不断学习思考的过程中,秉持求真、独立、自由和客观公正的态度,与社会对话交流,则可能产生一种巨大的力量。

在这充满迷雾的利奥塔式“独自漂流”中,踏上这条道路上的每一个人将从不同海港向不同方向出发,他的所作所为可能激不起任何一点水花,远方也没有一个确定的彼岸;海上漂流的大部分时间是孤独的,只可与他人偶遇然后接着分离;所回馈社会的结果可能无人所知、毫无影响力,也很可能是千疮百孔、一无是处;既无名声也无实利;却回报给个人精神气质上的昂扬和真诚。何曾有时——人类只能靠名声和金钱才能寻得快乐呢?

虽无彼岸,却拥有潘多拉魔盒中未曾逃逸出的“希望”,那就是社会的改良,伦理的进步。它可能过于浪漫主义空想,带有相当乌托邦色彩,但是,在路上!在批判一切(包括批判自己),寻求社会改良的道路上,必然得到个人精神气质的昂扬和真诚。我们努力,若有一天,我们倦了,不必强求,那就返航。即使返航,这也不是悲剧,我们的漂流便是我们的回报,一种尼采所说的“形而上学的慰藉”(参见\cref{chap:nicai})。其实我们所回去的港口无论如何也不会是我们早先出发的那一条港口了。这种“自下而上”的方式,或许真的有可能聚沙成塔呢。

这种不足同样也是明显的,个人民科理论、积淀、时间均不足,所提供的理论很可能漏洞百出、毫无价值,甚至未能达到所希望的客观,徒增他人耻笑。但是,这在主观上可以确据拥有当代所欠缺的一种\textbf{真诚},在一些方面比学界、组织拥有更强对于客观公正真实的诉求和勇气。由此而生的个人社会学不怕被批判,也愿意被批判。它的大胆不是为了作者个人博得名利,是为了社会能够因为批判而多些美好。它的宏大目的和意义不一定在于个人的成果,更可能是抛砖引玉,引发另一些探讨和批判,至少可以给其他个人社会学之路上的个人提供一些温度和勇气。即使此一人的个人社会学作品毫无价值,“只能交给老鼠的牙齿去批判”,那么彼一人的个人社会学或许能够更具普世意义,产生良好影响。伙伴们,切记,\textbf{求“真”}!,这是个人社会学唯一根本的优势,不可抛弃。

对当代社会的焦虑和解决方案的困窘必然催生出个人社会学这样的微观内容,如负责任却寻不到明确出路的当代社会专家们所提出的漂流、感受力\footnote{安东尼·吉登斯所著《社会学——批判的导论》}、想象力、城市革命\footnote{大卫·哈维所著《叛逆的城市》}、市民权利等同样带有这种微观、不确定和乌托邦性质。这类微观内容必然需要融聚以求达到宏观和明确,所谓个人社会学其实也包括对这些微观的融合。类似的思想,肯定有他人也在说合作,实在不需再去寻找论文引证,这是理论上的自信。

去探寻真实并积极回馈,体验蓬勃生命这“艺术的原始快乐”,做超人吧!
