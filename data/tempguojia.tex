
在中国,人们常常向往一位英明圣主横空出世,或者寄托于过去历史中的某位领袖人物。这
位圣主在为国为民的道路上,必然与旧的、压迫的、有损人民的强大体制做斗争。在这种艰
苦斗争的过程中,他披肝沥胆、远见卓识,虽数经艰险却屡屡化险为夷,实则是无往不胜,
他带领全国人民走向一个更为美好的图景。当其半道崩殂,继承他衣钵的人自然不会像他一
样有能力,甚至可能是一个反对他的反动派,这都使其巨大贡献受到折损甚至毁于一旦。

笔者在这里并不是说没有具有开创性和心怀天下的领袖,而是希望读者注意历史现实的限度,
这种限度使个人的或少数几人的能力是有限的,不管领袖如何英明,去违抗这种现实规律既
不明智也无法带来好的结果。当然,所带来的不好结果不是因领袖个人的主观或人性,而是
现实客观规律。

