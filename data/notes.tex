\chapter{笔记}
\label{cha:notes}

https://zhuanlan.zhihu.com/p/626981776

在他看来,空间生产是历史性的生产秩序:绝对空间生产—抽象空间生产—差异空间生
产。从绝对空间生产到抽象空间生产,显示空间生产不断走向物化和视觉化;从抽象空
间生产到差异空间生产,表明空间生产不断消除阶级冲突。

中国还有贫民窟吗?尽管中国城市化进程迅速,但答案是肯定的。

虽然许多西方观察家称赞中国的城市没有贫民窟,但现实情况要复杂得多。

在这篇文章中,我们将讨论“中国贫民窟”是否仍然存在的问题,如果存在,它们在哪里?

目录
[打开]
中国的贫民窟在哪里?
非正规住区分散在全国各地,特别是在大城市。

移民和低收入居民生活在不适当、不安全且往往是非法的住房条件下,无法获得基本服务。

一些城市试图升级这些地区或将居民搬迁到新的开发项目,例如上海,因为当地政府有资金这样做。

即使在今天,你仍然会看到上海的贫民窟遗迹。

If you decide to visit Shanghai someday, do not forget to check Central Perk Cafe!
上海贫民窟:

如果你想在他们完全消失之前看到,那就去北蔡靠近罗山高架,靠近北蔡地铁站,到达罗山高架路边。

北京贫民窟:

如果你想在它们完全消失之前看看,请前往朝外化石营1号Huashiying Unit。

中国人认为他们是贫民窟吗?
在中国,与贫民窟相关的是“城中村”,即现在已经破败不堪的老城区。

由于低收入居民的存在,有些人可能会将城市村庄误认为贫民窟,中国村庄与其他国家的贫民窟不同。

这些城中村,也称为“城中村”,存在于上海和广州等中国主要城市的郊区和市中心。

尽管被摩天大楼和交通基础设施等现代城市发展所包围,但它们仍保留着独特的乡村特色。

趣闻:
中国贫民窟
中国一些城中村居民希望自己的居住区能够被市政府选定进行土地规划,这样他们就可以在郊区获得几套新的公寓。

例如,在北京和上海等房价高的城市,如果他们卖掉一套公寓,他们就可以一夜之间成为百万富翁。

在中国,我们称这些从这种情况中受益的年轻人为“柴尔黛”拆二代,意思是他们的财富来自拆迁父母的蹩脚房屋。

贫民窟和城市村庄有什么区别?
中国贫民窟
贫民窟是通过快速的城市化和工业化形成的,而城市村庄曾经位于郊区,成为不断扩张的城市的一部分。
贫民窟是自愿和非法建造的房屋群,而城市村庄是合法的。
贫民窟的卫生覆盖率低,管理不善,而中国政府却努力改善城市村民的生活环境。
贫民窟往往是犯罪的温床,而中国的城市村庄是安全的。
农民工流动是由中国城乡就业推动的,而贫民窟居民往往遭受失业和极端贫困的困扰。
中国政府采取了哪些措施来解决这个问题?
中国贫民窟
中国已将城市村庄发展融入城市化进程,使其更加宜居。
国家推出了棚户区改造项目,发展了公共租赁住房部门,以改善流动人口和低收入家庭的生活条件。
最后的话:
截至2018年,近60%的中国人口是城市居民,而且这个数字还在继续增长。

该国对解决城市贫困和改善公民生活质量表现出坚定的承诺。

常见问题:
中国有无家可归者吗?
是的,中国有无家可归的人。

但是你不会像在西方国家那样经常在街上看到它们。

如果警察在街上发现他们,他们会先尝试联系他们的家人,90%的家人会接他们并觉得有义务照顾他们。

中国90%的人口住在哪里?
中国约60%的人口生活在城市地区,其余40%居住在农村地区。

然而,需要注意的是,中国是一个人口众多的国家,截至 2021 年人口超过 14 亿,因此即使是生活在农村地区的 10% 的人口也代表了非常多的人。

中国是否使人们摆脱了贫困?
是的,近几十年来,中国已经使大量人口摆脱了贫困。

根据中国政府的数据,2013年至2019年期间,近1亿人摆脱了贫困。

这是通过各种措施实现的,包括有针对性的减贫计划、基础设施投资和经济发展举措。

然而,值得注意的是,中国仍有许多人生活在贫困中或面临陷入贫困的风险,特别是在农村地区。


如果要对丁家庄城中村进行较为客观和全面的描述,必然需要警惕任何先入为主而未加证实的主观臆测。这些主观臆测常常受限于描述者个人所处的社会身份、个人经历与经验、意识形态取向等而落入片面、单薄、想当然的误区。而社会学定义了一个更讲求结构、组织和系统过程的人文科学研究框架,能够更为有效地避免或者减少这类错误认知。Warning: 是否可以引申出各个相关章节,做一简单说明。

在定量研究问卷调查中,因丁家庄城中村错综复杂的居住和人文环境,背景资料缺乏等方面的限制,所采用的样本抽样方式为(Warning:什么形式的)非概率抽样,这先天性的决定了样本的代表性程度可能偏低。同时,相对于全国成千上万个城中村,丁家庄城中村也只是其中特定的一个。这两方面缺陷注定本调查报告一些叙述可能过于特定或偶然,无法用以说明完整的丁家庄总体与其他城中村或棚户区。

本调查报告采用了摄影、定量与定性三者结合的方法,三者之间并不非简单隶属关系而更多是一种互补关系,相信这种结合能够有效增加调查的信度和效度。作为中国城中村来说,往往具有一些同质化的现象,本调查报告一些数据或者结论应当可以用于借鉴或参考这些现象。

丁家庄属于济南市姚家街道……地理位置……

调查理由:

丁家庄城中村是济南较大型且密集的城中村,在二十年前就已开始为外来务工人员提供住房等服务,Warning:本次调查中也碰到多家已在此居住二十多年的老租户,城中村一些问题较为突出和富有代表性。

丁家庄,又名丁家新村,其行政区域按照目前拆迁情况来看,大体可以分为南北两个部分。村南,位于工业南路以南,目前已动迁23家,尚有3家不同意拆迁。村北,位于华龙路以南,工业南路以北,包括不足千户居民,虽有部分居民已经签署拆迁同意书,但房屋尚未开始拆除。绝大部分房主家庭均未签订拆迁协议,拆迁政策目前也未有清晰说明。另外,在奥体西路已经在建丁家庄城中村改造安置房项目。这些将做而未做的事件有助于对新型城镇化规划影响下的丁家庄做一从前至后较为完整的梳理。

丁家庄城中村离我单位与宿舍较近,在正式社会调查之前我已多次在该地摄影,对路线和风土人情有基本了解,和一些居民建有良好关系,方便深入、频繁地调查。

根据《城市社会学——城市与城市生活》所说,如果我们只看到城市生活的各种事实,那么我们会错失城市生活动态的、充满活力的灵魂,城市也将显得无趣、昏暗而死气沉沉——由活动板房、电线杆、比比皆是的房屋出租广告等构成的集合…… 对于那些相当富有的人而言,城市生活常常就是一种经验,一种能够影响和实现他们自己的生活的经验(并且实际上也是一种能够影响他人生活的经验)。相反,对于贫穷的城市人,他们大多数是出于次要地位的少数种族和族群成员,城市生活则是一种必须竭尽全力来应对那些难以抗拒的力量的严酷之事。城市可以提供工作岗位和更好的医疗保健,娱乐教育等,以及促进技术和文化的进步,然而并非总是如此,更常见的情况则是,城市有时只会向特定的城市人群提供工作岗位和医疗保健。事实上,(setq configuration-layer--elpa-archives '(("melpa-cn" . "http://mirrors.tuna.tsinghua.edu.cn/elpa/melpa/") ("org-cn" . "http://mirrors.tuna.tsinghua.edu.cn/elpa/org/") ("gnu-cn" . "http://mirrors.tuna.tsinghua.edu.cn/elpa/gnu/")))在世界上绝大多数的发展中国家中,城市的情况都是有点铤而走险、孤注一掷的——并且在有些地方,情况甚至更糟糕。拉美、亚洲、非洲和中东……其中绝大多数城市都不能跟上这种人口迁入大潮,导致很多人陷入贫穷、营养不良和疾病。

\section{家庭}家庭成员组成,数量,教育、婚姻

\section{生活}节俭、衣着、娱乐


追随尼采的立场,“人性并不会在持续不断的斗争中逐渐进步,直到最后达到普遍的互惠,最终以法律准则取代战争;相反,人类将其每一种暴戾都深深地潜藏于法律体系之中,因而所谓人性的进步只不过是从一种统治形式过渡到了另一种统治形式而已“(Foucault,1977:151)


在福柯看来,像黑格尔或马克思等人所描绘的那种演化的历史,实际上是以一种非法手段,通过构造抽象概念体系而达到了其叙事的总体化的。造成的混乱比起所能解释的东西要多得多,它遮蔽了复杂的相互关系、分散变化的多元性、个别化的话语系列、各种不能被还原为某种单一规律、模式、统一体或纵向体系的事物。福柯的目标是打破那些巨大的统一体。

阶级斗争的二元模式,微观,微观欲望政治(micropolitics of desire)。试图通过解放欲望来促成彻底的改革。总体化、普遍化、力量哲学,奥斯维辛

欲望在本质上既不是善的也不是恶的,它只是动态的和生产性的。可以革命机器路线也可以法西斯主义机器路线。逃逸线既可以转变为解放之线,也可以转变为毁灭之线。

在他们对欲望的解释中,存在着一种本质化冲动与历史化冲动之间的紧张关系。他们没有想到,即使是欲望的多样性与生产性,也可能是在某种历史条件下形成的,有可能是明显的现代产物。有着本质主义的欲望概念。

福柯的工作也同样可以用来矫正鲍德里亚对内爆的分析。鲍德里亚断言所有的对立和分化界限都已内爆,而福柯却给我们展示了规诫和权力如何制造了隔离和分化,产生了等级制和边缘化,并对异己进行排斥。



本书反对那种认为本文与话语优于经验、感官及图像的文本主义看法,主张感官和经验优于抽象物和概念。在利奥塔看来,读是一种抽象的理性思维活动,寻求隐藏在符号(能指)下面的意义(或所指),而看却是一种直观的感性活动,它追求视觉的冲击而非隐含的意义。

”我们试图摧毁资本,并非因为它是非理性的,相反,恰恰正因为它是理性的。理性和权力乃是同一个东西。你也许可以用辩证的方法将一方粉饰起来……但你无法将另一方也粉饰起来,无法粉饰它的粗暴、监狱、禁忌、公共福利、社会淘汰、种族灭绝“。《driftworks》

在利奥塔看来,现代知识有三种状况:为使基础主义主张合法化而对元叙事的诉诸;作为合法化之必然后果的使非法化(delegitimition,宣布自身合法性的同时宣布异己话语、规则、标准、形式等为非法的做法)和排他;对同质化的认识论律令和道德法律令的欲求。

利奥塔置歧见于共识之上,置分歧和异议于一致和共识之上,置异质性和不可通约性于普遍性之上。

按照杰姆逊的说法,后现代主义代表了大量的文化变迁,其中包括:高雅文化和低级文化之间的坚固界限已告瓦解;现代主义作品受到了资本主义的完全认可和改变利用,丧失了批判和颠覆的棱角;文化几乎完全被商品化,从而失去了向资本主义发起挑战的批判距离……

拉克劳与墨菲认为,马克思主义展现出一种”一元论的渴望“,试图抓住历史的本质和深层意义,这种历史可以通过劳动和阶级斗争概念来理解,其逻辑具有铁一样的必然性,沿着一个严格的进化阶段序列自行演进。他们认为,马克思主义将复杂的社会现实简化成了生产和阶级问题。

他们建立了霸权概念的谱系学,揭示了它们在不同的历史情境中如何获得了各种不同的含义。尽管社会的日益分化碎裂说明了工人阶级之统一性的传统信仰是虚假不实的,但是,“霸权”却一直被用来围绕阶级概念将社会再整体化。因此可以说,霸权和本质主义逻辑是密不可分的,因为后者设想了隐藏在种种社会领域之后的深层本质,并将工人阶级本体化(ontologizes)为历史地真实地、普遍的主体。

拉克劳与墨菲在以下两个方面与马克思主义的社会主义观点进行了决裂:首先他们拒斥狭隘的“工人主义的”社会概念,反对……其次,他们拒斥社会主义的革命概念,反对把社会主义看成是同过去的千禧年式的大决裂。尖锐批判所有”国家主义“式的社会主义,指责他导致了科层化以及对个体的压制。按照他们的概念,社会主义并不是同资本主义过去的完全决裂,而是”民主革命的一个内在的发展阶段“(Hegemony and……:156)。

另外,正如培里·安德森(Perry Anderson)所指出的对后结构主义的批评(1984),话语理论倾向于彻底地摧毁因果观念,把历史和社会的确定性消解为随机性和不确定性。且拉克劳与墨菲抹杀了所有的政治力量之间的差别,将任何事物都等量齐观。

后现代政治可以看成是聚焦在认同政治和差异政治两面大旗之下。差异政治试图用被从前的现代政治所忽略的那些范畴(如种族、性别、性向等)来建立新的政治团体;认同政治则试图通过政治斗争和政治信仰来建立政治和文化认同,以此作为政治动员的基础。


在他们看来,当代社会组织形式乃是由资本主义与技术的结合构成的。不过,尽管批判理论家始终把资本主义视为一种生产方式,并且把生产方式是当前社会结构的重要构成要素,但是他们从不赞成任何形式的经济还原主义,不赞成把资本看成是决定社会构成和历史轨迹的唯一力量。由于他们采用了黑格尔——马克思主义辩证法的中介范畴和上层建筑的相对自主性概念,因而和正统马克思主义相反,他们承认国家、文化、各种社会制度以及个人的相对自主性。

批判理论认为,辩证法主要是用来描述不同社会现实领域之间(如经济与国家或文化之间的)关系的一种方法。中介(Vermittlung)首先是一个用于描绘某个既定现象的构造力量和联系的解释性范畴,而不是某种用以克服对立以便达成综合的魔术(如某些版本的黑格尔辩证法)。

他指出现代性的主流哲学观点植根于一种主体主义的意识哲学,与这种主体主义意识哲学相对立,他提出了一种以“交往行动”为基础的间主体哲学。哈贝马斯呼吁从意识哲学向交往哲学的“范式转变”。他认为自己所追求的这种转变,最早起源于弗雷格和维根特斯坦。但他声称语言哲学仍然太主体主义了,哲学模型仍然基于自我/客体模型,而不是立基于自我/他人的交往模型。

利奥塔, The Ethic of Care for the Self as a Practice of Freedom 在美学和艺术中,最重要的是对某种东西的亲身感受或者使别人感受到某种东西。




这个领域确实是天赋人权的真正伊甸园。那里占统治地位的只是自由、平等、所有权和
边沁\footnote{杰里米·边沁(Jeremy Bentham,公元1748年2月15日—公元1832年6月6日)是
  英国的法理学家、功利主义哲学家、经济学家和社会改革者。有用哲学即功利主义的
  代表人物之一。对他来说,个人的利益是一切行动的动力。然而,一切利益,如果正
  确加以理解,又处于内在的和谐状态中。各个人的正确理解的利益也就是社会的利
  益。}。自由!因为商品例如劳动力的买者和卖者,只取决于自己的自由意志。他们是
作为自由的、在法律上平等的人缔结契约的。契约是他们的意志借以得到共同的法律表
现的最后结果。平等!因为他们彼此只是作为商品占有者发生关系,用等价物交换等价
物。所有权!因为每一个人都只支配自己的东西。边沁!因为双方都只顾自己。使他们
连在一起并发生关系的唯一力量,是他们的利己心,是他们的特殊利益,是他们的私人
利益。正因为人人只顾自己,谁也不管别人,所以大家都是在事物的前定和谐下,或者
说,在全能的神的保佑下,完成着互惠互利、共同有益、全体有利的事业。

恩格斯指出:“根据唯物史观,历史过程中的决定性因素归根到底是现实生活的生产和再生产。无论马克思或我都从来没有肯定过比这更多的东西。如果有人在这里加以歪曲,说经济因素是唯一决定性的因素,那么他就是把这个命题变成毫无内容的、抽象的、荒诞无稽的空话。经济状况是基础,但是对历史斗争的进程发生影响并且在许多情况下主要是决定着这一斗争的形式的,还有上层建筑的各种因素。”经济条件归根到底制约着历史的发展,政治、法律、哲学、宗教、文学、艺术等发展既以经济发展为基础,又互相影响并对经济基础发生影响。并非只有经济状况才是原因,才是积极的,其余一切都不过是消极的结果。但这一切因素间的交互作用,“是在归根到底总是得到实现的经济必然性的基础上的互相作用”,“而在这种相互作用中归根到底是经济运动作为必然的东西通过无穷无尽的偶然事件……向前发展”。
