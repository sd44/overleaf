\chapter{前言}

\section{中国不能乱}

中国近几十年的发展有目共睹,我们的衣食住行得到了飞跃式的提高,坚硬的城乡二元
壁垒被打破,阶级流动前所未有,物质生活日益丰富,诸如此类。联合国也在多次报告
中提及,中国是为全球减贫作出最大贡献的国家。但同时我们也面临民族国家与全球化
等困境下的巨大挑战,人民不容易,国家也确实不容易。

资本主义危机避无可避、无法解决,只能转移来转移去。这一准则在民族国家与全球资
本化语境里的应用使世界政治竞技场愈加残酷。特别是在2008年世界经济危机之后,虽
然中国因强有力的国家管制未受它国那样损失——当然也存一些后遗症,但它国已从全
球化一端向民族国家一端加速倾斜,使中国的世界市场地位受到更大挑战。特朗普上台
后,更让美国倾向于经济民族国家立场,并用此来盘剥全球,中国这一世界市场、全工
业体系大国自然成为其最大对手。

笔者认为,习近平主席2017年所作重要论断“当今世界正处在百年未有之大变局”高屋
建瓴说明了当前世界的复杂局势,将历史、政治、经济、文学等科学融会贯通,涉
及1929年美国经济大萧条、民族国家与全球化、地缘政治、战争等方面,给出了重大警
示。

以阿根廷为例,当它面临秃鹫基金的绞杀时数次突围不得解,最终在以美国为首的部分
发达工业国家和世界金融集团压力下被迫高额偿债时,华尔街为阿根廷掉滴上一滴“善
良的”眼泪后立即弹冠相庆,华尔街鼓吹“从表面上看,似乎秃鹫基金是最大赢家。但
实际上,这起官司今天走到终点,对于阿根廷来说是个双赢的结果——\textbf{该国将最终得
  以重返国际资本市场},重新开展国际融资,以帮助经济走出困境。”\footnote{可参
  考\url{https://wallstreetcn.com/articles/230968}。}多么好的双赢啊!阿根廷发
行百年8\%的国债,比索崩溃,休克疗法等。阿根廷不再哭泣,它已经没有眼泪……

我们国家的一些金融从业者对华尔街的一系列操作热血沸腾,倾倒崇拜,欲让中国彻底
新自由主义化、全球化以有利于本集团利益。2020年10月24日,马云在上海第三届外滩
峰会上发表一番惊世骇俗的演讲,其深层目的在于欲让蚂蚁金服这一“科技公司”取代
央行货币的价值职能、流通、贮藏、支付手段;取代国家金融信用职能;消泯国家金融
监管职能,何其大胆!在其演讲行为和事后处罚力度上,我们都可以直观感受到金融资
本的强硬和野蛮。

在中美贸易战越演越烈时,有些人埋怨我们木秀于林导致困局,主张“投降”让利。羊
向狼送上自己的一块血肉,狼便会感恩退让么?持这样想法的人,幼稚或愚蠢到无以复
加。但也反映出我国在经济建设为中心的发展过程中,精神文明建设的失位。

国人常将国家作为坚不可摧,或可凭高层领导的英明步步走高,这是不科学的。国家面
临的是一个错综复杂实践中的现实社会,有些选择必须要面对,但这些选择的选项在一
些时候可能都是灰色的,无一较为完善,如薄一波在《若干重大决策与事件的回顾(上
下)》形容统购统销时所说“两种`炸药'中的选择”。\pagescite[][259]{boyibo}

中国不能再激荡,中国不能乱!有问题和分歧要探讨和争辩,可以声音大些但要理智,
尤其要综合全局。人之艰难、国之不易,上层下层均要互相考量和体谅。拥护党和国家
的领导,采用合适渠道、方法是前提,过度激进要不得。

中国体量太大,营养太丰富,国外反动势力意图阴谋颠覆中国之心未曾停歇。在当今残
酷的世界政治竞技场中,一个伤痛的中国将被诸多国家群拥而上,饕餮而食,渣骨不剩,
万劫不复……不了解民族国家与全球资本化下世界政治的这一残酷,要么是政治上幼稚
无知,要么是包藏祸心,望大家明鉴,莫要让亲者痛、仇者快。我们砥砺前行吧。

\section{本篇立意}

笔者已近不惑,越发感觉一切都是渺小的。从属于动物的人性给了世界历史社会太多限度,
世界历史社会给了国家太多限度,国家给了政府和国家领导人太多限度……而凡人是最为受
限的。不过任何层面总有些光明可以逃逸出重重限制,总有希望,总有多样可能。在这些丰
富的张力之下,若有一只上帝之眼去观看任何一人的一生,那都是深刻丰富和辉煌的悲喜剧
呢!

但是人类世界一直缺乏对社会现实正视和直面的勇气,常有一层含情脉脉的面纱蒙在真实和
悲苦之上,这层面纱或遮掩或美化或扭曲……

中国在数千年沧桑历史中形成的一种绝望使中国人的幻梦更为强大,很多国人常常呼唤一位
个人英雄,常常建立关于某个个人的偶像崇拜。在这种意识形态之下,我们常常
使“自己”缺场,抛弃了真实与悲剧之中的深刻、丰富和辉煌,隐患颇多。

% 为应对悲苦人生,尼采所说日神阿波罗的幻梦便长期占据主要地位,现代性理性对其它理性
% 的压制又使之愈演愈烈。

% 只有敢于直面悲苦,尊重生命本身,尊重人之所以为人,在纵情忘我的对生命原始冲动、创
% 造的追求中才能去求得真实满足和慰藉。\footnote{这其中也暗含一种悲观的可能——人是否
%   能大量摆脱动物性和原始性……}


% 笔者最高学历只为初中,只是人间一介草民。按理说,笔者各方面素养完全不足以较好完成
% 对中国方方面面的论述。那么为什么还要完成本篇?

% 相较知识、政治、经济界人,我的草民身份和个人特质使我更为敢言、直接,更想寻求客观
% 公正描述,更为注重凡人在时代波涛中的跌宕起伏。

民为重,社稷为轻、君次之。是的,我想写的是民的历史,而非人斗人的和具体个人的历
史。不管是社稷还是君臣、学者都应当成为民的历史的注解,而非相反。

在这种民的历史之下,我们将可以得到一个更为清晰醒目的认识,再用这种认识去造福于民,
寻得社会的真正进步。我的所作,将有诸多错误和不足。我之所以自不量力去从事历史描述,
因为我相信我的作品最重要的是激发大众和他人的思考,激发他人对我所作的批判,从而实
现独立之个人,自由之思想,光明之未来。即使拙作无人问津,毫无影响力,也已在这条道
路上投下一颗石子。

本篇着重对事不对人,还请理智对待。

% “关于中国”这部分着重于通过理清中国历史脉络以及笔者对当今中国问题的个人不成熟见
% 解来激发人的思考,从而减弱幻梦、尊重真实、勇敢面对、正视错误、科学发展、强盛国家、
% 保障人民以及为了人类的美好明天。这一目的并非有笔者亲自实现,实际上这系列文章应该
% 是几乎无人问津外加毫无影响力,但笔者可以成为立志于从事这方面的一系列推动者中的一
% 员。

\section{本篇文章结构}

本系列文章历史方面多为史料来源的总结和引用,个别方面加以真诚直接地辩证批判。因这个电子书项目是开源免费的个人项目,并无商业目的,应无侵权之忧,另外笔者会注明参考文献。之所以大量引用,出自于简明整理与普通读者难以去搜集整理这些资料的考虑。

第x章到第x章为总结当代中国从1949建国年至今的经济重大事件,第x章为总结说明和笔者对
当今中国的思考建议。本系列着重参考了薄一波《若干重大决策与事件的回顾(上下)》,
邓书杰,李梅,吴晓莉和苏继红所著《中国历史大事详解丛书(套装共9册
)》\cite{zhengshujie}\footnote{本套书为电子书,无法注明具体页码},blablablabla以
及对这些参考资料的追根溯源。



%%% Local Variables:
%%% mode: latex
%%% TeX-master: "../main"
%%% End:
