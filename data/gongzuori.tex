\chapter{草稿之一——工作制}

(Warning:医疗行业加班比较特殊,涉及国家社会保障方面,不是单纯的超长加班问题。我
对医疗行业完全不了解,没有办法写。有医疗相关从事人员或了解医疗行业的人可以与我联系。)

\section{传统超时加班企业}
在中国,劳动密集型企业是我们传统的超时加班重灾区,在制造业这种现象更为严
重\cite{guojilaogongshijian},曾经引发社会极大关注的富士康连续跳楼事件便被认为是
因超时加班所致,仅2010年一年被媒体曝光出来的就有“14连跳”。富士康为降低、杜绝此
类事件采取了一系列举措,2010年、2011年两次大幅提升工资、建全加班制,铺设大面积防
跳楼网,签订《不自杀协议》……2011年后,就很少有相关报道了。

根据《富士康工资、工时与生产管理调研》\cite{fushikangzuixin}一文,虽然至2015年为
止,富士康工人超时加班现象仍较严重,但大量裁员现象却与之并存。“2013年媒体又报道
了富士康新一轮的变相裁员浪潮(在2013年富士康全国用工规模减少了21万),引起工人
以`跳楼'”或停工等形式的激烈反抗。”。2016年,BBC发文报道《富士康用机器人取代
了6万名工人》。至于为何会发生超时加班与严重裁员并存的情况,我们放在下
面“\nameref{sec:gzryuanli}”一节中再做整理。

\section{新型超时加班企业}
金融和互联网企业是二十一世纪新加班重灾区。华为的“加班文化”流传甚广,有很多段子。
华为苏州研究院椅子背后常备的睡袋,酷爱加班文化的日本专家入职华为两个月后愤而辞
职“你们这样是不人道的”,华为总裁任正非挽留要回北京陪妻子的副总李玉琢时说“这样
的老婆你要她干什么?”,还有任正非对另起炉灶的李一男的“追杀令”最终迫使李一男重
回华为。华为在舶来词方面有个贡献,将日本“过劳死”的概念成功传播到了中国。

说起创意,华为也一点也不比富士康《不自杀协议》差。我国于2007年6月29日颁布《劳动合
同法》,其中第十四条规定“有下列情形之一,劳动者提出或者同意续订、订立劳动合同的,
除劳动者提出订立固定期限劳动合同外,应当订立无固定期限劳动合同:(一)劳动者在该
用人单位连续工作满十年的……”。这个法案将于2008年1月1日正式实施,在2007年底,华
为安排7000余名工龄8年以上的老工人向公司递交自请辞信,作为补偿,华为向这些老工人支
付约十亿元违约金,然后再重新聘用这些“失业工人”,工龄从零开始重新计算。数天
后,2008年1月1日,工人们“自愿”为了几万到十几万的补偿金,放弃了订立“无固定期限
合同”的可能,变为了1至3年劳动合同的新工人。“华为裁员门事件后,沃尔玛、环球、摩
托罗拉等公司也先后进行了应对新法的人力资源调整。《劳动合同法》实施前的“阵痛”似
不减反增。\cite{huaweimaiduan}

2010年8月,华为“公司14级以上工人被要求`自愿'签署《成为奋斗者申请书》……申请“华
为奋斗者”有一个必备条件,需要添加“我申请成为与公司共同奋斗的目标责任制工人,自
愿放弃带薪年休假、非指令性加班费和\textbf{陪产假}”这句话。\cite{huaweifendou}华
为真是个互联网企业的好模板。

任正非在2001年有篇文章《华为的冬天》非常有名,当时国内大公司似乎基本没有这
种“Winter is coming”的论调,更无一认为自己可能马上狗带。任正非的危机意识立刻被
广大媒体、企业称赞。近20年过去,在这种强烈的危机意识下,华为仍是高奏凯歌、一路前
行。而2017年,一些年过34岁的交付工程维护人员,过40岁的研发员工和45岁的老员工却可
能迎来了真正的凛冬,要被清退掉去他处过日子了。当然,华为已经惯例辟谣了。据说当前白色
家电企业也已经开始学习华为,对34岁工程师劝退……

2016年8月29日起,58公司总裁兼CEO姚劲波的微博陆续被众多58公司工人、工人家属和社会
人士浏览并情绪不稳定地评论\cite{tai58},起因是58公司在不发邮件和公文的情况下口头
传播了公司新工作制度——“996工作制”,早九点上班,晚九点下班,星期六正常上班,没有
任何补贴。虽然这种工作制可能并非58公司原创,但却是由因它开始引起公众反加班的社会
影响,并且在事件后,“996工作制”并没有受到实际影响,反而成为了一种明目张胆的制度。
笔者认为,58公司“996工作制”这一事件,可以定为中国劳动制度的一个里程碑。这标志着
严重超时工作制从原来只是个别公司内部隐性文化、不成文规定,发展至显性公司制度,并
最终成为一种公开的可以被任何企业复制的社会劳动制度。

滴滴出行于2016年底连发三篇大数据报告《2016年度加班最“狠”公司排行
榜》\cite{zuihen},涉及金融、互联网、公关、广告四个行业,包括33家公司。这33家公司
中最晚下班时间均在20点后,其中20点到21点之间下班的只有15家。四个行业相比较,工作
日时长方面金融行业较好,互联网业最差,最高加班时间是京东23点16分。周末上班方面,
金融业最差。0-5点下班返工人数方面,公关、广告公司表现突出,其中奥美广告返工人
数3080人,金宝大厦3家公司合计2102人。

\section{工作时长立法}

去看二十世纪或者当前的劳动法已经是难堪之事,那让我们粗略看下最早期的工时立法吧。

英国全行业立法是始于“1874年,R. A. Cross提出工厂方案,最终使得所有的英国工人都享
受10小时工作的权利”\pagescite[][96]{britishworkday}。

法国一步到位,直接是全行业立法,“法国1850年9月5日的十二小时工作日法令是临时政
府1848年3月2日法令的资产阶级化的翻版;这个法令适用于一切作
坊”\pagescite[][319]{capital}。

我们这些大体量的企业,靠着不懈的努力,向前150多年终于赶超到了英法19世纪中后期水平,
真是可歌可泣。另外,24小时工作制其实也已经来了,就是工作、睡觉、起床接着干活,一
天24小时不离公司,祝这些企业能早日赶超到19世纪早期劳动水平吧,就是不知道是否还需
要童工呢?

结合实际来看,我国现在实行的1995年劳动法,规定的8小时工作制相比其他各国标准较高,
要求劳动时长较短,比德国、新加坡等一周60小时工作制还要少不少。在实际操作上缺乏一
些空间,可能的解决方案如弹性工作制、休息权等问题还在探讨中。希望尽快出台相关政
策。

\section{超时加班的危害和原理}
\label{sec:gzryuanli}

对于超长劳动时间的论述和批判,首推马克思。希望大家能摆脱对马克思或褒或贬的刻板印
象,单从理论本身出发判断而不是意识形态的肯定或否定。

150多年前的《资本论》中提到的一些情况,居然仍高度适用我们的当前社会,在世界发展
方面来说这真不是一件庆幸之事。那么为何我们又重新回到150年前的工作状况?笔者认同大卫·哈维的说法——因新自由主义的盛行。

马克思原文如下(不想看马克思原文的读者,可以略过这部分):
\begin{quotation}

  至于个人受教育的时间,发展智力的时间,履行社会职能的时间,进行社交活动的
  时间,自由运用体力和智力的时间,以至于星期日的休息时间,——这全都是废话!但是,
  资本由于无限度地盲目追逐剩余劳动,像狼一般地贪求剩余劳动,不仅突破了工作日的道
  德极限,而且突破了工作日的纯粹身体的极限。\pagescite[][306]{capital}

  《1861年爱尔兰面包业委员会的报告》中提到,“委员会认为,把工作日延长
  到12小时以上,是横暴地侵犯工人的家庭生活和私人生活,这就侵犯一个男人的家庭,使
  他不能履行他作为一个儿子、兄弟、丈夫和父亲所应尽的家庭义务,以致造成道德上的非
  常不幸的后果。12小时以上的劳动会损害工人的健康,使他们早衰早死,因而造成工人家
  庭的不幸,恰好在最必要的时候,失去家长的照料和扶持。”\pagescite[][292]{capital}

  工人阶级中就业部分的过度劳动,扩大了它的后备军\footnote{想在本行业入职的失业或
    半失业人}的队伍,而后者通过竞争加在就业工人身上的增大的压力,又反过来迫使就业工
  人不得不从事过度劳动和听从资本的摆布。工人阶级的一部分从事过度劳动迫使它的另一部
  分无事可做(无事可做指后备军),反过来,它的一部分无事可做迫使他的另一部分从事过
  度劳动,这成了各个资本家致富的手段,同时又按照与社会积累的增进相适应的规模加速了
  产业后备军的生产。\pagescite[][733]{capital} 

  决定工资的一般变动的,不是工人人口绝对数量的变动,而是工人阶级分为现役军和后备
  军的比例的变动,是过剩人口相对量的增减,是过剩人口时而被吸收、时而又被游离的程
  度。\pagescite[][733]{capital}

  产业后备军在停滞和中等繁荣时期加压力于现役劳动军,在生产过剩和亢进时期又抑制现役
  劳动军的要求。所以,相对过剩人口是劳动供求规律借以运动的背景。它把这个规律的作用
  范围限制在绝对符合资本的剥削欲和统治欲的界限之内。”\pagescite[][736]{capital}

  不变资本的固定部分即工厂建筑物、机器等等的规糕,不管用来工作16小时,还是12小时,都
  会仍旧不变。工作日的延长并不要求在不变资本的这个最花钱的部分上有新的支出。此外,固
  定资本的价值,由此会在一个较短的周转期间系列中再生产出来,因而,这种资本为获得一定
  利润所必须预付的时间缩短了。因此,甚至在额外时间支付报酬,而且在一定限度内甚至比
  正常劳动时间支付较高报酬的情况下,工作日的延长都会提高利润。因此,现代工业制度下
  不断增长的增加固定资本的必要性,也就成了唯利是图的资本家延长工作日的一个主要动
  力。\pagescite[][91]{capital3} 

  资本主义生产方式按照它的矛盾的、对立的性质,还把浪费工人的生命和健康,压低工人的
  生存条件本身,看做不变资本使用上的节约,从而看做提高利润率的手
  段。\pagescite[][101]{capital3}
\end{quotation}

以上意思是指:
\begin{enumerate}
\item 工人、职员异化成为企业身上一个微小的、易损坏和易更换替代的一个器官。不止个
  人的健康、生命受到损害,他的社交角色、儿女角色、父母角色、夫妻角色等社会角色则
  被严重削弱和抑制。由此会引发什么问题呢?在中国传统的家庭人伦关系上,难以孝敬、
  赡养老人,难以增进夫妻关系,难以同子女产生良好沟通和教育,家庭组建、生儿育女的
  时间也都会滞后许多。尤其是华为《奋斗者申请书》居然明目张胆“自愿”要求放弃本就
  不多的陪产假,真是反基本社会人伦,千夫可指。当然,华为始终横眉冷对,淡定得很,
  人家负面新闻也像富士康一样越来越少了,这真是进步呵!

\item 在职工人、职员的过度劳动,使当前就业人数相对于正常劳动情况下应当就业的人数
  减少了。当前未就业或者半就业工人、职员的存在又使在职工人、员工的工资报酬被压
  低。

\item 随着生产力发展,机器等固定资本的支出愈加巨大,不管其使用或不适用,都在产生
  折耗,这是生产力发展所必要的。为了事实上节约这种固定资本,就需要延长工人劳动时
  间,这就使工人超长劳动似乎成为一种生产力发展的必须。另外,如果雇佣更多工人,则
  需要更多数量和更大投资的固定资本,这对本已庞大的不变资本支出无异于雪上加霜。

\item 此外,脱离马克思文本,就实际情况来看,通过超长工作日所产生的工人、职员数量
  节约,使企业培训成本、培养成本、管理成本、福利保障成本、风险成本均大幅降低。

\end{enumerate}

综上,企业倾向于让员工尽量加班而非在原工作时长不变的情况下雇佣更多工人。

那么工人、职员为什么要加班呢?(Warning:这里感觉还很单薄,希望读者能与我联系,
提供宝贵建议和意见。)
\begin{enumerate}

\item 如富士康这类工厂,将基本工资设的较低,工人的工资财富积累常常只能在加班费中
  来完成。不加班,不赚钱。加班了,才有钱。这是两百来年很多大工厂惯用不变的伎俩。
  或者如华为将部分加班费融合进了工资,一些职员由此认为工资是可观的。这两种情况其
  实都是一样的,就是企业确实补贴了部分工资。但工人、职员本应得的远比实际得到的其
  实要多的多。

\item 如华为这类企业,已在社会具有相当的名声。大众普遍认为能进华为的,肯定是有一
  定水平的。能在华为干住的,跑到其他企业肯定是不怕累的。在一些职员看来,被压榨就
  压榨吧,加班就加班吧,一个跳板而已。撑过现在,未来是光明的。

\item 企业考核。不管是领导思想上的还是落在纸上的考核,都将加班量作为职员是否合格
  的一个标准,华为有个词很有意思,叫“工作量饱满”。工作量不饱满,你就不是个称职
  的员工。工作量饱满了,员工的个人时间,家庭时间也就别想饱满了。

\end{enumerate}

员工作为这个微小的、易损坏的、易更换替代的工厂、企业小器官,在被榨干后,甚至只是
在自己年龄增大后、工龄福利增多后,就将迎来自己被扫地出门的结局,他们将被年龄更小,
福利待遇更低的年轻员工所取代!超长工时所带来的利润诱惑是极易在工厂、企业间传染的。
这便是自由!实则是资本的自由,而非人的自由!

\section{工时改革问题的民族国家与全球化困局}

重新结合实际情况立法,对工人、工会赋权,加入弹性工作制和阶梯性休假等补偿措施,加
大执法力度,主动执法,对较重违规现象进行惩罚性罚款等,这些举措理论上将能有效解决
工时严重过长的问题。

就以往传统历史经验来说,19世纪英国工作日改革经验表明,“劳动生产率和劳动强度的变
化,或者是在工作日缩短以前,或者是紧接着在工作日缩短以后发生
的”\pagescite[][601]{capital},后来福特8小时工作制以及世界上广泛传播的劳动法均在
一定程度上证明了这点。

但现实是严酷的,我们可能无法去实行这些举措。因为立法要考量社会整体在立法后的整体
得失均衡。如果只是工人工时得到大的改善,导致其他方面造成矛盾激化,对国家、企业甚
至工人自己产生比立法之前更为恶劣影响的话,就不符合科学立法的精神。笔者认为,传统
工时改革的成功依赖于国家政治地理上的整体性、一致性和封闭性,一旦颁布条令就覆盖全
国工厂、企业,工厂难以迁至他国,必须接受当前法律条令制约。而当前有一条横贯在工时
改革前的巨大矛盾就是,全球化对世界上各个民族国家均造成巨大压力,而中国工时改革必
将更为扩大这种压力。

安东尼·吉登斯在《现代性的后果》一书中提出,现代性的全球扩展趋向产生了一个“失控的
世界“,它的出现没有人也没有政府能够全面地控制。马克思用怪物来描述现代性,而吉登
斯将其比作坐在巨型汽车或”猛兽“上面。笔者认同吉登斯上面所言。所谓全球化,说穿了
就是资本在全球高速通畅的流动。不管是中国政府,还是美国政府,不管是工人还是大企业
家在面对全球化这一庞然大物时都常是有心乏力的。

全球化给中国工时改革,甚至是劳动改革所带来的巨大压力,主要表现为两个方面,
一个是东南亚、南亚、拉美劳动市场的低廉——相当程度上是因为贫困人口、童工、更贫穷国
家的劳动力输入,和法律不健全等引起;另一个是美国降低税收,大力吸纳国际企业和本国
企业进驻,制造业进驻的政策扶持力度也一直在扩大。以富士康\cite{foxconnwiki}为例,
其在巴西、匈牙利、斯洛伐克、土耳其、捷克、日本、马来西亚、墨西哥、印度、美国均有
工厂。印度方面已与富士康签订谅解备忘录,富士康预计在5年之内投资印度50亿美元,“美
国方面唐纳德·特朗普总统于2017年7月26日宣布,富士康将在威斯康星州东南部建立一个价
值100亿美元的平板电视制造工厂。威斯康星州将每年向富士康支付高达2.5亿美元的补贴,
为期十五年。这笔交易被一些人批评为是拿取30亿美元纳税人税收资助激励富士康。威斯康
星州立法机关的无党派预算办公室的分析确定,国家纳税人将在2043年收回投资。”

如果中国工时改革成功,工人劳动时间趋于正常,相关产业领头羊们很可能不会将重心放在
另想它法,提高劳动生产率上,以解决因工时恢复正常所带来的效益下降——而是转战早就已
经或者正在海外建设起来的基地。因企业实力不足、性质和特点限制而走不出去或难以走出
去的企业们则可能承担来自全球化国际市场的巨大竞争压力。届时中国除大规模资本逃离外,
将出现过多被产业抛离出来的,难以就业的真正剩余劳动力,造成社会问题。

此外,福耀玻璃董事长曹德旺宣称美国实际税收远比国内实际税收低廉。美国政府也配合其
支持取缔工会。这是美国政府给其提供的“胡萝卜”,但是他却没说“大棒”——美国政府为
吸引他国生产制造业等资本密集型工业进驻美国国内,所提出的关税、反垄断、反倾销威胁。
(Warning: 这方面希望读者提供更多有力确实资料)。这也是全球化面前,民族国家之间
的矛盾。

全球化给工时改革带来如此巨大压力,那么如何破局?如何尽可能的完善?笔者能力弱小,
所能做的可能只是提出以上问题了。国家目前已经越来越关注“构建和谐劳动关系”,让我
们拭目以待吧。

最后,用诗人许立志的一首诗来结束本节吧。

\poemtitle*{流水线上的兵马俑}
\settowidth{\versewidth}{Than Tycho Brahe, or Erra Pater:}
\begin{verse}[\versewidth]
  沿线站着\\
  \qquad 夏丘\ \qquad \quad  张子凤\qquad  肖朋\\
  \qquad 李孝定 \qquad  唐秀猛\qquad  雷兰娇\\
  \qquad 许立志 \qquad  朱正武\qquad \\
  \qquad 潘霞\  \qquad \quad 苒雪梅\\
  这些不分昼夜的打工者 \qquad 穿戴好\\
  \qquad  静电衣 \qquad  静电帽  \quad  静电鞋\\
  \qquad  静电手套 \quad 静电环 \\
  整装待发 \quad 静候军令\\
  只一响铃功夫\\
  悉数回到秦朝
\end{verse}
\newcommand{\attrib}[1]{%
  \nopagebreak{\raggedleft\small #1\par}}
\attrib{许立志 (1990--2014)}


%%% Local Variables:
%%% mode: latex
%%% TeX-master: "../main"
%%% End:
