\chapter{序言}
\label{chap:preface}

我出生于山东省一个财政困难县,幼时幸得父母、长辈宠爱有加,家中虽不富裕却不缺
衣短穿,也因此不谙世事,不解人间种种,总活在自我的小天地中。等到初中二年级时
便遭恶果反噬,世间一切对我来说太过未知和魔幻。尼采所述人类社会那怒吼着“你应
该”的巨龙形象,让我只敢蒙着头瑟瑟发抖于被窝之中无所适从。这世界究竟是什么样
子?为什么会这样?我该怎么做?一切皆未知……

2014年,我所在单位和宿舍均搬至济南市化纤厂路,步行不足200米便是丁家庄菜市场。
约莫过了两年,我才注意到菜市场后面的一个中大型城中村——丁家庄城中村。我听说
过、见过贫困的农户,却未曾想到城市城中村的人文环境竟如此奇特,便在此租了一个
单间,以求近距离了解。两个月租期到期后,也常漫步于丁家庄。

人已经在场,可无知如我怎样去深入了解这样一大风土人情呢?深圳的邱文建议我采用
社会学的视角去观察丁家庄,并多次对我鼓励和指导。我接受了他的建议,学习社会学
知识,继而延伸到历史、政治经济学后,本电子书便立项了,此书也可理解为我个人三
观的阶段性总结。

然而,并没有什么终点或者结局……

致力于人类共同体发展的大师、宏观叙事一片寂静。马克思的学说可分为三部分:经典
哲学部分,因其只满足于认识世界、而对改造世界有先天缺陷,早被马克思自己扬
弃。“一切社会变迁和政治变革的终极原因……不应当到有关时代的哲学中去寻找,而
应当到有关时代的经济中去寻找”,马克思由此走向政治经济学,足够深刻地揭示了资
本主义的破坏性;但在建设性上只有对“自由人的联合王国”“自由王国”等的希翼,
乏善可陈,实则充满绝望。科学社会主义部分只有纲领而缺少具体措施。

在20世纪50年代之初的西方,左翼思想由强盛急转为衰颓,1968年法国、日本、美国、
西德、意大利、墨西哥、巴西等各国“未曾被预见,也不可预知”地同时爆发以学生青
年为主力的左翼运动,然而这些运动紧接着就被更强大的右翼浪潮压倒……

宏观叙事衰颓后,深受尼采和马克思影响的后现代主义理论又发扬起来,认识到了现代
社会理性“可以吞噬一切”。它反对总体论、本质论和霸权,倡导微观的欲望、解辖域
化、少数族群权利及多元化。后现代主义者们起初就能意识到现代理性的强大统御异化
能力,于悲观和批判中求彼岸。但不久之后便被虚无思想所占据,后来转出一批虚无主
义者。这时其实已经不能叫做后现代主义了,它没有了批判精神。其实疲弱的当代人哪
有什么勇气去信奉虚无主义!万物皆虚,那唯有自己的切实体验为真。 所谓当代虚无主
义不过是个人功利主义的一张难看画皮而已。

在城中村改造上,本书主要参考列裴伏尔、大卫・哈维等人的空间生产理论。在历史上,
主要论述了苏联建国起至上世纪三十年代初,以及中国部分时间段的政治经济思想史。
此外还有些零散内容……

一切的这些对我的个人生活似乎并没有什么用处。抽象、一般的社会理论知识与个人具
体的现实实践之间有太多太多的不同,指导不了多少东西。即使理论竟也是如此贫瘠。
政治经济学、批判社会学式微;当代哲学不必说改造世界,就连对认识当前世界似也已
提不起兴趣;就如何去理解人类,理解善恶,日本一些人文社科作品似乎提供了一个角
度——礼赞万物为生而做的一切努力和挣扎,慕强而轻善恶,但这样也就在相当程度上
将人性与兽性等同,强力大于善恶,我想这也是日本至今仍在滋生万恶军国主义思想的
原因;如今我想要求得心灵更多平静,儒释道这类心性学问倒是挺适合,哈哈,真是未
曾想过的道路……

本书框架草稿于2019年完成时,绝望情绪不减反增,只不过个人能以更好心态去应对绝
望,写作之路就此终止。其实当我走向丁家庄、去写这样一本费时费力又各方面不讨好
的电子书时,何尝不是我熟用了二十余年的懦弱手段——逃避,逃避虽可耻却有用……

恍惚间又过五年,来到了2024,我已过四十岁,各方面机能均开始走下坡路,也无意再
去搜寻什么大道理,已经开始“认命”,接受自己的各种不足,一切均是自己种下的果,
果再酸却也不必强求什么。此书似乎更无完成必要,也找不到什么理由。

但我确实很喜欢这本书,它无用且丑陋,但终归是我的存在。如果今时我不去修订完成,
那这本书就再也完成不了啦。好吧,随意吧,那就写完吧。如果要给此书找唯一的优点,
那便是真诚。

尼采曾盛赞一位夫人,因其对自己的孩子说“亲爱的,你总是做傻事,做傻事让你特别
快乐”。我想我也是这样一个傻小孩吧,哈哈……

感谢丁家庄愿意信任我并接受社会调查的人们。感谢深圳的邱文,他使我踏上社会学的
学习道路,并多次对我鼓励和指导;成都的saintjoe,多次与我就摄影和社会展开探讨。
感谢知乎和微信上的到芬兰车站群组,特别是杜若致远、Exphilonous、行止、寒霜雪蝶
等人,你们的洞见和激烈争论使我获益良多。


最要感谢的是我的妻子康利。结婚数年来,我屡屡不务正业,不事生产,总做些傻事惹
人耻笑,妻受苦受累颇多,却无几分怨言,静静包容我的一切胡闹。

谨以此书献给我的两个孩子,子墨和子韩。希望书中有些内容可以使你们在将来少走些
弯路,多些对社会的认识和理解。


% 但韩康利虽不% 支持但也不反对,事实上纵容我傻乐,并默默承担起家中老人与两个孩子的繁重照顾工作。% 此生我最大的幸运是娶到了韩康利,因这一幸运我不再有任何一点立场指责上天待我凉薄。
% 焦虑、质疑、批判是人类进步的必要要素。不能认可批判的建设性价值,将面对停滞、
% 倒退或向更为危险的境地发展。批判要建立在对真实和事实的求证基础上,它不止有
% 批评,同样包括对好方面的肯定。我们生活在一个资本现代性的理性愈加强大的时代,
% 它强大和贪心到试图去规训,并能够在相当程度上规训其它一切理性,妄图称王使所
% 有理性臣服。这更需要我们具有批判精神。

% 这本电子书是个四不像但又什么都有的怪物,涉及政治、社科、经济、人文等诸多方面,
% 贯穿始终的核心是笔者自身应用社会学对于社会和人\improve{以后加入资本?}的思考。
% 笔者并非学富五车的专业人士,自身缺陷与恶习比比皆是,这均使本书内容存在各种各
% 样缺陷,惹人耻笑。它也几乎不会产生任何影响力或激起什么波澜。但我想它最重要和
% 最强大的意义,是展现出一种个人真诚、直接和积极的社会学历程(详情请
% 见\cref{chap:gerenshehuixue}),一些人也正自觉或不自觉地走在这样的道路上。笔
% 者愿作这条道路上的一块铺路石,以使同行者不那么孤单、寂寞。这条道路的发展将使
% 人类和社会受益。

% 十多年前,笔者在尼采某本书\footnote{年代久远,我已忘了这本书的名字和具体内容,只是粗略记忆。}中看到某夫人对她的儿子说“”时被击中了。笔者想做好事,想把好事做好,但总阴差阳错的走向相反面,许是无奈之下只好从这许多傻事中吸取快乐了。十多年过去,笔者已近不惑之年,还是一直在做傻事,平添父母、妻子、子女的烦恼,或许写作这本书只不过是笔者所做的又一件傻事而已,整件事只是笔者自我逃避的又一个山丘……


% 最要感谢的是我的妻子韩康利。结婚八年来,% 我不务正业,不事生产,屡屡做些傻事惹人耻笑,没有经济建树又身在外地。但韩康利虽不% 支持但也不反对,事实上纵容我傻乐,并默默承担起家中老人与两个孩子的繁重照顾工作。% 此生我最大的幸运是娶到了韩康利,因这一幸运我不再有任何一点立场指责上天待我凉薄。


% 笔者所见所知并无多少创新、发展,尤其与一些富有强烈人文关怀的专业社科人士相比,我% 简直像幼儿园小朋友一样所知甚少。但这此书最主要的目的和意义,并非是书中蕴含多少真% 知灼见,而是在书外。在于本书读者几何,影响力% 多少,甚至有多少错误其实已经不重要了。变,那就是自我抒发我个人对于社会、国家和政% 治的见解。至于

% 个人朝三暮四,贪乐误事,丁家庄这个项目常常面临夭折和停滞,其目的和方向也总是% 一片混沌。

% 在国际政治的角斗场中,如果中国因为社会剧痛剧变从而变得贫弱多病,就必然被许多国家% 暂时搁置他们彼此之间的争斗,转头立刻联合起来对中国群起攻之,分而食之,进而敲骨吸% 髓,让中国万劫不复。看不到这一点的人,在政治上是幼稚无知或者是别有用心的。之所以% 会造成这种局势,并非是因我们中国和外国的政治体制或意识形态不同,实质上我认为我们% 与他国的共同点——即使在政体和意识形态上——远比不同点多的多,甚至呈多倍比例。只是因% 为中国物产丰富,疆域辽阔,市场第一,发展态势迅猛及潜力巨大,且在历史上我们同欧美% 各国的交流,没有他们彼此之间的互通交流频繁。我要声明,我坚决拥护党和国家的领导,% 对所谓西方自由民主等陷阱和中国重改良等坚决排斥。但是我坚持轻改良,所谓正能量,不% 是不说难听的话、不指出事物缺陷,而是使事物向更好方向发展的力量,改良是需要的。

% 在这过程中,在项目进行过程中,喜闻有两位姑娘,可能是大学生也在进行丁家庄的社会学% 问卷调查工作,同意接受调查的村民或租户可以获得50元奖励。我感觉可能是国立大学调查,% 很是欣喜,希望中国能在社科方面快速发展。

% 如果有可能的话,我想做的下一个项目是不像丁家庄项目这么宏大难控的,它的表达方式更% 加具体和容易引起社会影响,同时也更加注重感官而不是理性,可以更多借助摄影等方式,% 那就是关于ADHD(小儿多动症)的批判。小儿多动症是一个症候群,其中一些表现可以归为% 个人特质或非精神性病变等,国外已有多部相关著作,深度理论构建应当可以从反精神病学% 与精神病批判上汲取养分(即使没有深度理论,项目应当也可获得成功)。学习社科这一年% 多来,我已经有爱上社科了,但就现实情况来看,丁家庄项目是我进行的第一个社科项目,% 也很可能是最后一个。我的年龄、基础、精力、家庭、经济能力等似乎很难允许我再做类似% 的工作。如果有读者能进行ADHD这个项目就太好了。

% 特别感谢泰安的李玉刚,与我进行多次最接地气的激烈讨论并直言不讳。特别% 感谢丁家庄愿意信任我并接受调查的人们。% 感谢微信上广州HiFi_Tam所建摄影群,其中北京刘烜超,广州Hifi_Tam,广州邱邱,南通兽% 无不摄,上海Keith,厦门Resean均对本项目不成气,但我也无力去更改的摄影部份提出宝% 贵意见(以上排名均安拼音顺序,不分先后)。


% 谨以此书献给我的妻子韩康利。

% 2014年,笔者的工作和住宿地点均变更为济南市化纤厂路,距离丁家庄菜市场不% 足500米。2016年7月份,笔者为练习街头摄影常在街头漫无目的地游荡,有次穿过菜市场忽% 然发现了一片新景象——丁家新村,济南人俗称丁家庄,这是一个城中村。笔者虽已在济南工% 作9年,在丁家庄附近生活2年,也去过济南市诸多地方,但从不知道有丁家庄城中村这样一% 个存在。

% 虽然笔者就出生和成长在一个贫困县,也去过几次农村,但初入丁家庄仍因它表面的破败和% 杂乱而产生一种恐惧感,总感觉可能有治安危险。它既不同于城市也不同于农村,一些地方% 甚至比农村还要显得困窘残破。

% 如何不流于表面地展示这里?如何避免消费苦难,去真正深入地表现这个地方?笔者二三十% 次进入丁家庄城中村,仍未找到答案。生活在深圳的邱文向我提了一个建议,用人类学的眼% 光去拍摄、组织照片,直接展示他们的衣食住行、教育娱乐等方面,使其呈现出丁家庄城中% 村的整体生活面貌,并可作为一个样本进行留存。

% 笔者在邱文的建议的基础上,最终选择了社会学方向来做丁家庄的全面考察。“空间生% 产”这一块比较知名的学者有列斐伏尔、大卫·哈维等人,这些学者基本都是批判实践社会学% 方向。批判社会学的奠基人一般被认为是马克思,列斐伏尔、大卫·哈维等人也深受马克思主% 义影响,要较好理解他们的著作,难以跳过马克思。因此我又学习了《资本论》等马克思、% 恩格斯著作,这使本书带有马克思色彩。

% 社会学学习的进展,和笔者摄影水平的粗浅,笔者渐感有必要% 随着学习的深入,也因笔者摄影技术的粗浅,笔者渐感摄影的无力,% 当代世界,不管是左派还是右派,大多数人对马克思的理解常常是肤浅甚至错误,“马克% 思”往往成为一种标签和符号,一些人看到这个标签和符号就产生强烈的好恶感和价值判定。% 希望大家能够从理论内容本身来判定,而非这种先入为主。

% 保有随着拍摄练习的深入,我对丁家庄开% 始关注起来,想将丁家庄城中村作为一个摄影项目来做。在进行过程中,项目的施行方式和% 角度也始终在变化,从起初单纯个人的摄影项目,发展到对丁家庄城中村的了解欲望,并进% 一步发展至对中国新型城镇化规划的关注,最终发展至对中国发展的态势见解。

%%% Local Variables:
%%% mode: latex
%%% TeX-master: "../main"
%%% End:
