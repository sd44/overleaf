\chapter{城中村的简单概念}

所谓城中村者,城市和村居性质兼而有之:它往往位于城市中心或郊区,作为城市的一
部分,周边均具城市特征,自身却充斥着农村式的无序和自然,缺乏人工的总体规划,
各家各户的宅地界限比传统村庄混乱得多,基础设施(能源、通讯、供水、交通、安全、
卫生、医疗、文化等)薄弱,常生人口基本为农村户籍,土地制度仍为农村集体所有制
而非城市的全民所有制;作为农村,它的外来流动人口数量数倍,甚至数十倍于常住人
口,耕地被大量或完全占用,转为商业或住宅地产,耕地的这种性质转变使常住人口原
赖以生存的农业收入转为地产收入,并成为其收入的重要来源。

2016年《联合国人居三筹备委员会第三届会议政策文件10:住房政策》(文号A
/CONF.226/PC.3/23)对贫民窟词条所作的解释为:
\begin{quotation}
  自 2003 年起,联合国会员国商定将\textbf{贫民窟家庭}界定为生活在同一屋檐下,但缺乏
  以下五项条件中的一项或多项的一组个人:(a) 能得到经改善的饮水;(b) 用得上经
  改进的卫生设施;(c) 充足的居住面积-不过于拥挤;(d) 住宅的结构质量/持久
  性;(e) 土地保有权的保障。
\end{quotation}

我们通常所说的城中村可以认为是中国政府棚改文件中定义的“城市棚户区”,并且明
显属于联合国定义的城市贫民窟范畴。\cite{unandchina}


中国的城市贫民窟人口有多少呢?联合国人居署提供的往年数据如下:
% Please add the following required packages to your document preamble: % \usepackage{booktabs}
\begin{table}[!ht] \centering
\label{my-label}%
  \caption{1990-2014年中国城市贫民窟人口比例及数量}
  \capsource{联合国人居署旗舰报告《World Cites Report 2016》\cite{9789211327083}}
   \adjustbox{width=\linewidth}{%
    \begin{tabular}{l|l l l l l l}
      \toprule
      & 1990年 & 1995年 & 2000年 & 2005年 & 2010年 & 2014年 \\ \midrule
      城市中贫民窟人口的比例(\%) & 43.6 & 40.5 & 37.3 & 32.9 & 29.1 & 25.2 \\
      城市贫民窟人口的数量 & 1.316亿 & 1.514亿 & 1.691亿 & 1,835亿 & 1.806亿 & 1.911亿 \\ \bottomrule
    \end{tabular}%
  }
  \end{table}

  根据《国家新型城镇化规划(2014-2020年)》,我国预计“到2020年基本完成城市棚
  户区改造任务”。根据2018年第十三届全国人大一次会议政府报
  告\footnote{\url{http://www.gov.cn/gongbao/content/2018/content_5286356.htm}},
  “棚户区住房改造2600多万套,农村危房改造1700多万户,上亿人喜迁新居”。全国
  轰轰烈烈兴起的城中村改造跑步前进,新型城镇化取得了惊人成绩,这同时也标志着
  曾经遍布每个大中型城市的老式城中村的大量消亡,丁家庄城中村也在其中。先
  让我们看下曾经的丁家庄是什么样子吧。

  \chapter{济南市丁家庄城中村见闻散记}

  \section{背景介绍}

  济南市丁家庄,又名丁家村、丁家新村,据1992年5月1日所立村碑记载:
  \begin{quotation}
    明永乐年间(1403-1424)当地根据传说取村名“定妖庄”。后因此名不雅,故
    以“定”字谐音“丁”字改为丁家庄。
  \end{quotation}

  丁家庄隶属于山东省济南市姚家街道,曾是济南市一个较大且密集的城中村,
  在2000年前就已开始为外来务工人员提供住房等服务,共有村民宅基地(院落)
  近800户5000人,外来流动人口峰值大约可达30000人。丁家村城中改造是山东省棚改
  旧改的重点项目,于2017年年底基本完成房屋拆除工作,拆迁面积约为53万平方米,
  包括村民宅基地、公益性公共设施用地和经营性用地等,整体搬迁至奥体西路新建高
  层小区。

  \section{初见}

  笔者初入丁家庄时便因它表面的破败和杂乱而产生一种恐惧感。整个城中村
  除20余栋6层楼房以外,基本全是村民自建、层层加盖的三四层楼房,有的自建房已有
  些轻微倾斜,有的在房顶上再加装简易活动板房;各种样式的电线、网线、不明用途
  的线缆随意聚成一团团,与敞盖或无盖的配电箱、歪七扭八的电线杆纠结交织,这里
  似乎随时会演变为危房倒塌或大型火灾或现场。笔者倒是未听闻未见相关事故,或许
  是因户主租户有一套自发自治管理的办法。

  条条未经规划和硬化的水泥石板路路面也是蜿蜒曲折,纵横交错整个村落,笔者游历
  丁家庄数次之后才可不迷路。出租房基本都是单间,十几二十几户共用户主搭建的公
  共厕所,楼上住户冬天起夜时还要穿衣下楼,并不方便。村民们多是传统农民打扮,
  外来务工人员衣着也称不上光鲜亮丽。过往中国农村生活条件艰苦,重病患者、残障、
  丧失劳动能力的比例往往大于城市,而丁家庄这里的比例可能比传统农村还要高。笔
  者有次刚要走出丁家庄时碰见村口一位50岁左右的男子坐在轮椅上斜着头,面无表情、
  眼神空洞,对周围不管不问地在晒太阳,或许是偏瘫。未过几秒,迎面又走来一个怯
  生生的30岁左右的男子。他提着午饭低头走来,看见我时便将整个身子直接旋转180度,
  定在原地,用后背对着我,不敢和我有一瞥眼的接触。当我正要和他擦肩而过时,这
  位男子又朝无人那侧180度急速转身,继续前行,他应有视线恐惧症或社交恐惧症等精
  神疾病吧。不过有熟识的村民也向他热情问候。\improve{缺乏对居住人口的描写}

  未接触过城中村的人,初入丁家庄,很难不恐惧吧,毕竟这里像是随时随地会发生刑
  事案件一样。然而,这确实多虑了。房东们多会查看并登记租户身份信息,考察租户
  人品性格,周边人群间的关系也比高楼大厦上来得亲密一些。一只只小小的、防君子
  不防小人的普通挂锁足以保证财物安全。不知怎的,当我在那不足10平的单间里住宿
  时,却比在现代小区高楼上居住更加平静和坦然一些。

  \section{黑夜与清晨}

  丁家庄城中村的夜总比周边来的更早些。

  街道路灯不多,住户多使用散发着黄色光晕的白炽灯。晚上八九点钟,灯光和墙面组
  成的黄色主色调混合着个别店铺的七彩霓虹灯光闪耀在城中村里,叮叮当当的做饭声
  时常响起,笔者还见过住处一楼楼廊里一位不足10岁的小男孩独立炒菜做饭,归来的
  叔叔阿姨们在路过时对他不吝赞叹。

  深夜,村外尚有较晚收摊的小吃车、大排档、夏天24小时营业的烧烤店、为深夜食客
  们服务的小零售店、长明的路灯和过往的车辆。村里却是另一片景象,这里更黑更静。
  晚10点左右还能偶尔听见晚归人家的锅碗瓢盆交响曲,11点左右整个村子便一下子寂
  静起来,水泥石板路上鲜有路人。偶在没有路灯照耀的环卫点,有一位老人在几个垃圾箱
  中翻找可再利用或可卖的杂物。

  这里的早晨也比周边来的更早些,6、7点钟各处雄鸡打鸣,各家各样的声音均透过不
  隔音的墙壁和窗户传到家家户户,问候声、寒暄声也此起彼伏。孩子、送孩子上学的
  家长、上班族一下子散布各处,城中村在这个时间已经开始繁忙起来。

  \section{医疗保健和社会保障}

  因笔者能力有限、怠惰和调查时间选择上的问题,被调查人群多为中老年人并且数量
  很少。虽然存在这种样本偏差导致不能推导出一般性的结论,但笔者认为可以提下自
  己直观和经验感受:不管是流动人口还是常住人口,他们身高较之周边明显偏矮,心
  脑疾病较多,也有被调查人家庭两代人中均患重大疾病的事例。《中国心血管病报
  告2017》中开篇有提“我国居民心血管病(CVD)危险因素普遍暴露,呈现在低龄化、低
  收入群体中快速增长及个体聚集趋势。”,贫穷始终是种顽疾,甚至是绝症。

  根据“丁家庄环境卫生管理公示牌”,丁家庄有保洁人员21人,保洁面积4万平方米。
  据本次调查,济南市环保局贯彻执行八小时工作制,并为保洁人员缴纳三险。所聘用
  保洁人员多为丁家庄居住人口,每月到手收入在1600元左右。济南市环保局在劳动保
  障上的表现出乎笔者意料,在此点赞。另外,有一例保洁员工伤纠纷,当事人为外地
  来济60多岁老人,因是否算工伤与环保局有分歧,环保局领导也曾亲切慰问。虽然当
  时问题并没有解决,但当事人对国家和政府仍表示“非常满意”,没有任何意见。

  丁家庄老年村民大多没有缴纳任何形式的养老保险,包括新型农村社会养老保险(新
  农保),也不了解具体政策。满60岁老人由村委每月补助600元左右。丁家庄大部份村
  民所能缴纳的社保只有新型农村合作医疗(新农合),有每年缴纳100元和300元两个
  档次。新农合在丁家庄村民重大疾病治疗上发挥了极其重要和显著的作用,常可报
  销60\% 多的费用。

  \section{何不食肉糜}

  虽然丁家庄北侧就是一个较大的综合市场,供周边几个社区的居民采购农副产品,生
  意兴隆但年租金昂贵,大菜摊租金约9万元/年。城中村里仍有些蔬果摊子长期固定在
  街道一角,它们铺设在水泥道路或机动三轮车上;还有些定期定时流动叫卖的轻型贩
  菜货车。城中村蔬果摊主要面向城中村内中老年居民,无需承担任何租金,多供应次
  一级的菜品并且价格低廉,但销售额和利润远不能于综合市场相比。 (可参
  考\cref{fig:caishichang}, \cref{fig:caitandajie1},
  \cref{fig:caitandajie2}。)

  这里我们来谈一个架设在机动三轮车上的水果摊主吧,笔者将其隐去名讳,代称为刘
  哥吧。即使在城中村,刘哥家的居住环境可能仍是最为糟糕的,他一家三代住在一
  个200多元月租的单间中。他的立业史,也是每一步都恰被时代所驱赶和压迫的悲剧史。
  他做过走街串巷流动叫卖的小贩,被驱逐淘汰;又做过居民小区外较固定的摊贩,被
  驱逐淘汰;又做过丁家庄综合市场外路边摆摊的摊贩,相较综合市场的高额租金,路
  边摊所需费用便宜太多,同样也是被驱逐淘汰;最后刘哥成为了城中村内一个机动三
  轮车上的瓜果小贩。2017年丁家庄已被夷为平地,刘哥不知又将以何种方式去拖家带
  口继续书写他自己的立业史……

  夏季的一天,笔者碰到他十岁左右的儿子从自家中捧来几片薄切的西瓜给他吃,可
  他明明就是卖瓜的呀。

  笔者无论如何也没有想到,当笔者向一些人说起丁家庄综合市场和城中村菜摊的区别
  时,有些人会指责刘哥不努力不争气,质疑他为何不早在综合市场租摊位(高租金)
  以求得良好收益。一个一家三代居住在城中村破败单间的外来人,如何去承担每年数
  万的租金啊。我记得90年代初当我还是个小孩时,并没有铺天盖地地听到“成
  功”“成功人士”这类说法,对于贫穷者多半是怜悯其境遇不佳。今时简单用资本结
  果作为个人能力的衡量标准,从而一叶障目时,我们都将成为那个呆傻可笑的晋惠
  帝,“百姓无粟米充饥,何不食肉糜?”


  \section{过往的宅田基地之殇}

  % “社会空间” 概念。他把社会空间理解为 “空间的实践”、 “空间的表象” 和
  % “具象的% 空间”的三位一体。也就是说,空间不是一个抽象的存在,而是自然的、精神的和社会的三位% 一体,这个一元的整体性的社会空间既是逻辑-认识论的空间,同时也是社会实践的空间和亲身% 经历包括感觉想象的空间。通过这个包含着诸多规定和关系的社会空间概念, “社会空间” % 概念。他把社会空间理解为 “空间的实践”、 “空间的表象” 和 “具象的空间”的三位% 一体。也就是说,空间不是一个抽象的存在,而是自然的、精神的和社会的三位一体,这个一元% 的整体性的社会空间既是逻辑-认识论的空间,同时也是社会实践的空间和亲身经历包括感觉% 想象的空间。通过这个包含着诸多规定和关系的社会空间概念,

  % 列斐伏尔强调,三元分析的三个方面是既相互独立又相互作用的,三重要素中每一个方面都可% 以也应该在与其他两个方面中的任意一个的相互关系中得到理解。并且任何一个要素对于其% 他两个都不具有绝对的优先地位。

  % 社会空间是 “空间的实践” (spatial practice)、 “空间的表象”(representations
  % of % space)和 “具象的空间” (representational spaces)的三位一体。社会空间的生产可以从% 这三重概念的辩证运动中得到理解。相应的,如果我们从身体出发理解社会空间,这三重图式% 又可以表达为 “感知的” (the perceived)、“概念的”(the conceived)和 “经验% 的”(the lived)。


  % “空间的实践”
  % 包括生产和再生产,以及构成每一社会的典型的具体场所和空间化% 的位置。一个社会的空间的实践通过保证连续性和凝聚力,生产并隐藏这一社会的空% 间。

  % “空间的表象” 是概念化的空间。

  % “具象的空间” 既是作为 “居住者”、“使用者” 或 “占用者” 的人们生活于其中的空% 间,包括房屋、教堂、广场等;也是一些艺术家、哲学家、作家们所描述但又渴望不仅% 仅是描述的空间。因此,一方面,具象的空间与人的真实的生活经验相连,与空间的表

  在中国传统小农经济的农村中,往往以家庭为基本单位,主要活动范围局限在村内,
  生产集中在自家规模极为有限的耕田或从事简单手工业、半加工业的家中,生活集中
  在自家住宅与村内公共空间。田区和住宅区常常分隔明显,呈大块状分布,同一大块
  内常是多家彼此有联结的田地或住宅,其中相联两家之间的田地多用沟、垄、界石作
  为田界,住宅多用共用的一面墙壁或距离极近的两面墙壁作为宅界。

  多数人的社会空间长期固定、聚合、封闭在居住村,物质和社会资源有限,生产生活
  单调贫乏,村民之间联系频繁,信息传播速度快,这种情况下发生的利益冲突常常尖
  锐持久、难以调和,田界和宅界作为最重要的家庭产权,具有相当刚性。

  在实践和具体的社会空间中,这一刚性界限却又常常变动并受到侵蚀。它本身包含农
  村中通风、采光、日照、排水、通道等难以界定的方面,另外在国家和政府层面来说,
  又有历史遗留、立法不健全、执法成本高等问题;在村民来说,则有历史遗留、普遍
  违规超额占用、法律维权成本高、法制观念不强、宗族势力等问题。

  如果氏族大家庭或直系小家庭被其他大小家庭侵犯界限而未采取有效措施,则不单是
  家庭经济效益,连带个人和家庭的自我认同、社会地位也将受到严重的负面影响。这
  也使宅、田基地矛盾相当尖锐频发,家庭中的强壮男子往往被赋予保卫甚至扩张这一
  刚性界限的责任。过往中国对家庭伦理的重视,甚至重男轻女等现象,也多由此建立
  起来。

  在笔者对丁家庄的走访过程中,也曾碰到宅基地纠纷当事人商姐说起一例二十多年前
  的惨剧:两家因新修墙壁越界而产生的宅基地矛盾步步升级,致使受到屈辱的商姐丈
  夫自杀。商姐带领家人将邻居群殴至半死,法院支持了邻居的索赔诉求,但商家一家
  并未履行法院判决,于是肇事方二十余年不允许商姐家开发自家一块空地。

  丈夫自杀后,商姐带着两个儿子独立生活,其中小儿子当年不到两岁。她抓住了空间
  转化的时机,是90年代丁家庄第一批建立租屋的人。丁家庄拆迁前她的租屋总面积已
  经超过1500余平,出租房屋超过50户,月入过万,在丁家庄这也是了不起的成就。她
  的两个儿子也争气,可给她带去一些宽慰,只是长年的劳苦使其腿病明显,面态老相,
  但商姐勤劳不改,晚上仍会去丁家庄综合市场进行清洁工作,以换取一些额外的微薄
  报酬。总是风风火火、穿着男式西装、一瘸一拐地支撑起整个家庭的商姐啊……

  当商姐以维权为理由倾诉她的故事时,笔者一再说明我这个自发小项目不能给其带来
  任何改变,也难以根据她的一家之言去支持她时,她对笔者的中立态度表示认可,最
  后甚至还埋怨起她的家庭当年为什么不让一下那寸土的界限,即使再让更多些也不要
  紧啊……

  诉说过程中,笔者渐感商姐并非真要维权,并非还存那样的恨意。她曾说要带笔者去
  看那块因久被搁置而自然成为停车场的空地,笔者当时就感觉可能不需要了,事实上
  也是如此,事后我为求证自己想法向她两三次询问时间安排时,她果然都借故推脱,
  并刻意回避我。

  她真正想要的其实只不过是一场没有利害关系的倾诉……倾诉完之后当事人产生了羞
  愧感,不敢面对被倾诉人;而所谓仇恨早已经在时空的变换中变成了一介难以抹去的
  难看疮疤。

  随着丁家庄旧村改造的完成,丁家庄人将进入新的、现代化的城市空间,城市空间中
  的界限更为明显也易维权,原本农村中典型的宅基地矛盾将很少存在,希望两家日
  后能够相忘于江湖,也同样希望人们彼此能够多一些理解,多给一些倾诉的空间,以
  使悲剧不再那么多,那么难以承受……



  % 丁家庄城中村是一个充斥着盎然生机、孕育着诸多可能的城中村。它是许多人实现梦想的起% 步点或中转站,也展开怀抱接纳了各方边缘人群,诸多住户之间较为和谐。它也远未成为一% 个堕落之处,这里并非治安恶化严重,


  % \section{失败的丁家庄城中村社会学调查}

  % 笔者起初准备以定量研究方法结合定性研究方法对丁家庄作一个较为全面的社会调查。其中定量方面,仿照ISSP\footnote{The International Social Survey Programme,国际社会调查方案}和中国的CGSS\footnote{Chinese General Social Survey,中国综合社会调查,于2007年加入ISSP。}问卷作一个针对丁家庄城中村的调查问卷,定性研究初步决定采用Phil Francis Carspecken的批判定性研究框架。

  % 之后笔者又十几次进入丁家庄,也曾在丁家庄居住做过1个月进行田野调查,但因自身怠惰和三心二意使本次调查大失败,不过以下几点心得或许有益于类似社会调查的开展,在此分享给读者:
  % \begin{enumerate}

  % \item 定性研究一般要求记录谈话,录音常是记录谈话的主要方式。扎根理论和Carspecken的定性框架必须建立在大量谈话资料的多次整理上。但丁家庄人均谈录音色变,拒绝录音。这种拒绝主因是被调查人在敏感性的事件上害怕录音成为某种对自己不利的证据。

  % \item 城中村人员组成和住房结构的复杂,使针对整体的概率抽样问卷调查非常困难。实际上笔者认为,只有具有政府背景的组织或机构大力支持、推动才能完成类似复杂区域定量研究的概率抽样。

  % \item 笔者采用了非概率的随机街头访问方式发放调查问卷,这使调查结果可能产生无效的、完全不具代表性的样本。并且即使如此,问卷回收率仍然极低,只能勉强算是10\%,使定量分析成为不可能。

  % \item 利益敏感问题信度不高。除笔者本人能力拙劣外,也有现实客观。例如针对房东的调查中,房东往往隐瞒和减少实际出租房屋间数以及出租收入,可从被调查房东所处房子建筑外观、体积以及丁家庄出租房的平均面积和收入得出这一信度不高的结论;针对所有人的收入问题也存信度问题,再三询问或试探所得出的收入结论最高浮动为1000多元人民币。调查问卷中针对村委和拆迁方案的满意程度采用了5级李克特量表,但被调查人极端选择较多,情绪化明显,个人利益最大化主导的倾向明显。

  % \item 半数以上房东有对上级政府机构的强烈诉求,这也是他们对社会调查人最大的期望。本次调查为笔者个人自发,没有任何组织机构背景,也一再向被调查人言明本人所写报告预计不会产生任何一点社会影响力,无法满足房东这种诉求。除此之外,社会学可以采取小额金钱奖励的方式来增加被调查人积极性,但笔者着实囊中羞涩,无法采用这种方式。

  % \item 租房人对本次社会调查表现出严重的整体冷漠,可以认为这是一种社会排斥。关于这方面内容,笔者放在之后章节再详细论述。小额金钱奖励应可以有效提高租房人积极性,但未实施,原因同上。

  % \item 最主要原因仍是笔者个人社会调查能力的欠缺,和态度的不端正。笔者接触社会学是在丁家庄摄影项目受阻之后从零开始,在社会学意义上的与人交往也存各种缺陷。最主要的还是态度,三心二意、半途而废,甚至因屡次消沉而遗失了几份已经回收的完整调查问卷。

  % \end{enumerate}

  % 本次调查过程中,笔者听闻有两位女士几乎同期在丁家庄城中村进行社会问卷调查,并对问卷完成者提供每人50元奖励,效果不错。笔者估计是具有政府背景的组织机构,如大学在做这份工作。希望我国能够在当前基础上进一步普及社会学调查相关知识,增加社会学调查项目,并保证社会学调查的中立性及公信度,同时也希望调查者能够坚守信度和效度问题。

  % 下一节笔者将介绍在丁家庄中的所见所闻所感。
