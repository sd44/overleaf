\url{https://www.thepaper.cn/newsDetail_forward_11744994}


非常重要!!!!!!!!!!!!!
!!\textbf{https://www.aisixiang.com/data/138457.html}
\textbf{王小鲁:土地财政的昨天、今天和明天}
1998年,全国国有土地出让成交价款约500亿元,相当于当年地方财政预算收入的10%。此后土地出让收入以不可抑制的高速度膨胀——2001年1300亿元,接近地方财政收入的17%;2011年3.3万亿元,相当于地方财政收入的63%;2021年8.7万亿元,相当于地方财政收入的78%(均不包括在地方财政预算内)。

从2001到2021这20年,现价GDP年均增长12.4%,现价全国财政预算收入年均增长13.4%,而土地出让收入年均增长23.4%,远超过了经济增长和财政收入增长(据财政部、国土资源部、国家统计局公布数据计算)。

计算平均地价,2001年每公顷土地出让收入143万元,2011年为978万元,2021年为2393万元。20年间地价上涨15.7倍。

同期消费价格指数(CPI)只上涨58%,工业生产者价格指数(PPI)只上涨38%(据财政部、国土资源部、国家统计局、华经产业研究院数据计算)。

过高的地价大幅度推高了房价,并改变了国民收入分配格局。由于不包括地价和房价,消费者价格指数(CPI)和生产者价格指数(PPI)已不足以反映真实的通货膨胀率。

而地价和房价飞速上涨,是几方面影响因素合成的结果。

从供给侧来说,一个正常因素是城市规模经济的溢出效应带来的城市用地和城市周边待开发土地升值。

从需求侧而言,城市化快速发展造成的房地产需求上升,而土地资源相对有限,持续拉动了房价和地价上涨。

第三个因素的作用可能更加突出,即长期以来货币增长大幅度快于经济增长,货币超发造成的多余购买力大量流向房地产,造成越买越涨、越涨越买的趋势,成了长期不破的资产泡沫。

第四个重要影响因素是地方政府的作用。对地方政府来说,土地出让收入来得容易,用得方便,监管不严,透明度低,一方面可以用来搞各种地方政府想搞但没有其他资金来源的投资项目,既包括城市基础设施和其他公共建设,也包括政府楼堂馆所等各种自我服务设施,另一方面还可以增加包括“三公消费”在内的党政机关行政管理费用。

更严重的是,在政府垄断地源的情况下,大量的官商勾结腐败案件都与土地和房地产交易有关,给腐败官员带来了巨额收入,造就了一大批隐性富豪——在种种利益驱动之下,地方政府可以利用其独占地位,通过控制土地投放量,借“招拍挂”尽量抬高地价,以获得最大收益。这也导致政府在资源配置中所占份额不断扩大,挤压了市场配置资源的份额,形成与改革开放前面20年的放权让利截然相反的趋势。

※※※※※※※※※※※※※※※※※※※※※※※※※※


反面论述
https://www.aisixiang.com/data/115819.html
赵燕菁:从土地金融到土地财政:资本的胜利、有为的政府与城市的转型

※※※※※※※※※※※※※※※※※※※※※※※※※※
https://bfi.uchicago.edu/wp-content/uploads/2022/02/BFI_WP_2022-24.pdf

文献中的典型观点是,住宅用地出让主要是地方政府增加收入的一种方式,而工业用地出卖主要是为了补贴产业、刺激经济增长、支持劳动力需求。

中国的土地市场具有相当大的工业折扣:工业区用地比住宅用地便宜一个数量级。与以产业补贴或促进产业增长为中心的解释相反,我们强调了未来土地税收的重要性,并发现地方公共财政激励措施可以在很大程度上合理化这种价格差距。在“土地融资”制度下,土地出让是中国地方政府的重要收入来源。研究表明,在中国,地方政府作为垄断性土地销售者,面临着住宅用地或工业用地供应之间的权衡,这取决于工业和住宅用地销售收入的不同时间分布、地方政府的财政约束以及地方政府与其他各级政府分享税收的程度。

公司税收收入和土地出让收入;2019年,这两个数字分别约为8.7万亿元人民币和7.3万亿元人民币.2工业用地产生持续的未来税收流动,因为工业企业缴纳增值税和所得税,以及各种费用。由于中国没有住宅物业税,住宅用地销售只会暂时增加房屋开发商缴纳的税款.3这意味着地方政府面临着一个选择,即出售前期收入较大的住宅用地和出售工业用地,因为工业用地支付的税收现金流比实际收入更持久。

这种动态的观点意味着,大量的前期工业用地折扣并不一定意味着政府正在通过廉价土地系统地补贴工业。事实上,我们表明,在调整住宅开发商缴纳的税款后,来自工业用地的税收流动可以定量补偿前期工业用地折扣。我们还提供了地方政府的融资需求影响土地分区的因果证据,表明地方公共财政在通过土地分配渠道塑造中国经济增长路径方面发挥着被低估的作用

我们从一个概念框架开始,分析推动工业用地而不是住宅用地供应均衡回报的力量。我们考虑的是地方政府,其目标是最大化其财政收入的现值。除了全部属于地方政府的前期土地销售收入外,住宅用地还产生了由房屋开发商支付的一次性税款,而工业用地则产生了工业税的持续现金流,并与中央政府共享。在均衡状态下,由于未来的税收优惠,当地政府愿意以较低的价格出售工业用地。该框架指出了两个简单且可衡量的汇总统计数据。首先是工业折扣,即工业用地和住宅用地之间的价格差异。第二种是工业用地销售的内部收益率(IRR),计算为贴现率,折现率等于工业和住宅用地销售的所有现金流的现值

※※※※※※※※※※※※※※※※※※※※※※※※※※

就像许多其他国家一样,中国也有严格的分区限制。正如Chen等人(2018)所强调的那样,划为住宅用途的土地的售价大约比划为工业用途的土地高出十倍。2019年,中国住宅用地平均价格为3,619元/平方米,工业用地平均价格为304元/平方米。我们将住宅用地和工业用地之间的这种价格差异称为工业用地折扣(或工业用地互换)。

※※※※※※※※※※※※※※※※※※※※※※※※※※
地方政府从土地中获取融资有两条直接渠道:一是土地财政,即通过出让国有土地(主要是商服用地和住宅用地)50至70年的使用权来获取级差地租;二是土地金融,即将土地注入地方融资平台来撬动资金为城市建设融资。

已有研究表明,土地财政和土地金融相结合的以地融资模式催生出一个高效的融资体系,极大推动了近二十年来的中国经济增长,特别是与房地产和基建相关(包括钢铁、水泥)的重工业部门得到了飞速发展,居住环境和基础设施的改善又进一步促进了地价和房价的升值,为下一轮以地融资创造了有利条件。这样的正向反馈机制使得中国自1998年以来一直处于投资驱动型的增长阶段。

如果说曾经的以地融资模式主要依赖土地财政,那么如今,随着部分地区土地资源的告罄和征地补偿标准的提升,一些地方政府对土地财政的依赖开始减退,而对土地金融却愈加青睐。由表1可知,越是土地资源稀缺的地区(东部地区),土地抵押贷款规模越是高于土地出让规模,而对北上广这样的一线城市而言,土地抵押与土地出让的比值更是高于东部地区的平均水平。

图1显示,2003至2010年间,出让成本占比基本维持在六成左右,但在2011至2018年间,无论是土地出让成本的规模还是其占土地出让毛收入的比重都呈加速上升趋势,甚至到了2018年,土地出让成本占土地出让毛收入比重已接近九成。

从征地规模的角度亦可佐证土地出让成本在2010年前后出现了飘升的现象。图2显示,农用地和总的土地征收规模在2011年之前处于上升区间,但在2011年后呈逐年加速下滑的态势,这恰好是土地出让成本开始飘升的年份(参见图1)。迅速缩小的征地规模表明,未来国有土地的出让规模将大幅缩减,利用土地财政进行创收的空间将被进一步压缩。

1994年的税收分享改革,该改革减少了地方政府在许多税收收入来源中的份额,同时并没有减少他们的支出需求。要回答“为什么土地金融在中国(到目前为止)有效运作”要复杂得多。在这方面,了解其促进土地金融体系盈利能力的制度基础是很重要的。首先,地方政府被指定为地方城市LUR的垄断提供者。因此,当一个城市提出购买农村土地并将其转换为城市土地时,它不会面临来自其他实体的竞争。此外,支付的补偿是基于当前(农业)用途的土地价值,而不是未来城市用途。地方政府回购、分配或出售的城市土地利用率并将其转换为新的土地利用率时,也适用类似的规定。这种安排使城市政府能够获利,直到最近,土地价格的暂时上升趋势加剧了这一点。值得注意的是,支撑土地金融体系的丰富制度框架既有法律、官方文件等显性规则,也有在实践中具有影响力的隐性规范。其中一些细节在现有文献中似乎没有得到适当的承认,并可能导致对该系统的误解。希望我们的分析能够更清楚地说明该系统在中国联邦体系中的运作方式

自1980年代以来,中央和地方政府在中国经济中扮演的角色已经很明确——中央政府负责规划和监管,而地方政府则负责在地方层面实施这些法规。

从1980年到2020年,地方政府在国家一般公共预算中的份额从45.7%增加到85.7%,而同期收入在一般公共预算中的份额从75.5%下降到55.5%。为了弥补收支缺口,地方政府依靠中央政府补贴、债务融资和土地融资。

1994年实行的税制改革在随后的几年中对土地融资产生了巨大的溢出效应。改革规定,中央政府将把所有与土地有关的税收分配给地方政府。这包括土地价值税、城市土地使用税、耕地占用税、土地增值税、契税(财产转让税)和一般财产税。更重要的是,土地出让的所有利润都转移到了地方政府。

这一新制度为地方政府增加了土地出让收入提供了强大的动力。从2000年到2020年,地方政府土地出让收入占总收入的比重从5.9%提高到42%。如果将其他相关的土地税也考虑在内,土地收入占地方政府总收入的一半以上(52%)。

o“通过信用制度,未来的收益可以贴现到今天,使得资本的形成方式得以摆脱对过去积累依赖,转向预期收益。”“中国城市伟大成就背后的真正秘密,就是创造性地发展出一套将土地作为信用基础的制度——‘土地财政’。”


土地财政是指政府出让国有土地使用权收入,及与土地有关的税收如耕地占用税、土地增值税等,这都是有法律、制度依据的。土地金融是指政府拿土地做抵押,向银行贷款,属于政府负债。这样做,没有法律依据,是打了制度的“擦边球”,因为制度允许政府作为国有土地所有权的代表经营土地。
