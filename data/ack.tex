\begin{acknowledgement}

  在我对丁家庄城中村的摄影项目止步不前时,深圳的邱文提出转用人类学眼光来直接拍摄的建议。没有他的建议,我恐难选择社会学并步步走来写就这本电子书。在此犹表感谢。
  
  成都的saintjoe,南通的兽无不摄,广州的HiFi\_Tam所建的摄影群在摄影项目进行过程中,始终对我给予帮助、批评和指导,在此表示感谢。

  在我拍摄与调查丁家庄过程中,丁家庄常住与流动住民,始终表达出惊人的善意和信任,没有你们,这个项目既不可能开始也不可能完成。我辜负你们了,丁家庄项目总是半途而废,最终难以成形。
  
  知乎的阿迦陀针对维基解密一章提出宝贵建议,木语针对当代艺术一章提出宝贵建议,河里的谢军、xieyuyan2019提供了宝贵的文献参考资料并提出个人看法,感谢你们。

  泰安的李玉刚,与我进行多次最接地气的激烈讨论并直言不讳,感谢。

  最为感谢的是我的父母、妻子、一儿一女,我这辈子不事生产、不务正业、烟酒不离,屡屡做些傻事惹人耻笑,对家几无贡献,又常年不在家中与你们相伴。明知愧疚却无所作为。
  
  谨以此书献给我的妻子韩康利,家庭是我童年、少年时的大幸运,遇见你是我成年后最大的幸运,这种幸运之巨大使我没有任何理由去埋怨上天不公,上天是眷顾我的呢!而遇见我却可能是你最大的不幸。

  \rightline{孙滨}
\end{acknowledgement}
