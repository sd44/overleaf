\chapter{前言}

\section{百年未有之大变局}

中国近几十年的发展有目共睹,我们的衣食住行得到了飞跃式的提高,坚硬的城乡二元
壁垒被打破,阶级流动前所未有,物质生活日益丰富,诸如此类。联合国也在多次报告
中提及,中国是为全球减贫作出最大贡献的国家。但同时我们也面临民族国家与全球化
等困境下的巨大挑战,人民不容易,国家也确实不容易。

资本主义危机避无可避、无法解决,只能转移来转移去。这一准则在民族国家与全球资
本化语境里的应用使世界政治竞技场愈加残酷。特别是在2008年世界经济危机之后,虽
然中国因强有力的国家管制未受它国那样损失——当然也存一些后遗症,但它国已从全
球化一端向民族国家一端加速倾斜,使中国的世界市场地位受到更大挑战。特朗普上台
后,更让美国倾向于经济民族国家立场,并用此来盘剥全球,中国这一世界市场、全工
业体系大国自然成为其最大对手。

笔者认为,习近平主席2017年所作重要论断“当今世界正处在百年未有之大变局”高屋
建瓴说明了当前世界的复杂局势,将历史、政治、经济、文学等科学融会贯通,涉
及1929年美国经济大萧条、民族国家与全球化、地缘政治、战争等方面,给出了重大警
示。

以阿根廷为例,当它面临秃鹫基金的绞杀时数次突围不得解,最终在以美国为首的部分
发达工业国家和世界金融集团压力下被迫高额偿债时,华尔街为阿根廷掉滴上一滴“善
良的”眼泪后立即弹冠相庆,华尔街鼓吹“从表面上看,似乎秃鹫基金是最大赢家。但
实际上,这起官司今天走到终点,对于阿根廷来说是个双赢的结果——\textbf{该国将最终得
  以重返国际资本市场},重新开展国际融资,以帮助经济走出困境。”\footnote{可参
  考\url{https://wallstreetcn.com/articles/230968}。}多么好的双赢啊!阿根廷发
行百年8\%的国债,比索崩溃,休克疗法等。阿根廷不再哭泣,它已经没有眼泪……

我们国家的一些金融从业者对华尔街的一系列操作热血沸腾,倾倒崇拜,欲让中国彻底
新自由主义化、全球化以有利于本集团利益。读者们应该比较熟悉日本和韩
国“财阀”一词,其实财阀即“康采恩”垄断集团。康采恩以其金融机构为核心,直接
或间接控制多行业领域中的企业,往往涉及各行业全链条或主要部分,从而获取高额利
润。

中国也有自己的康采恩,如腾X系和阿X系等。2020年10月24日,马云在上海第三届外滩
峰会上发表一番惊世骇俗的演讲,其深层目的在于欲让蚂蚁金服这一“科技公司”取代
央行货币的五大职能——价值尺度、流通、贮藏、支付手段;欲要超高杠杆、无节制发
展金融信用;消泯国家金融监管职能,何其大胆!

2021年4月10日,市场监管总局依法对阿里巴巴集团处以行政处罚,并处以182.28亿元人
民币。2021年7月7日,中国金融监管机构对蚂蚁集团及旗下机构处以71.23亿元罚款,并
关停旗下的“相互宝”业务。两次巨额罚款之后,阿里股票大涨,资本市场认为其“利
空出尽”,信心回升……在其演讲和事后处罚上,我们都可以直观感受到金融资本的野
蛮狂妄和强势地位。


在中美贸易战越演越烈时,有些人埋怨我们木秀于林导致困局,主张“投降”让利。羊
多上几分怯懦卑微,向狼送上自己的一块血肉,狼便会感恩退让么?只是该怎样应对,
确实是一个重大难题。

中国体量太大,营养太丰富,国外反动势力意图阴谋颠覆中国之心未曾停歇。在当今残
酷的世界政治竞技场中,一个伤痛的中国将被诸多国家、甚至本国金融精英群拥而上,
饕餮而食,渣骨不剩,万劫不复……不了解民族国家与全球资本化下世界政治的这一残
酷,要么是政治上幼稚无知,要么是包藏祸心,望大家明鉴,莫要让亲者痛、仇者快。
我们砥砺前行吧。

\section{本篇立意}

笔者已过不惑,越发感觉一切都是渺小的。仍具相当动物性和原始性的人给了世界历史
社会太多限制,世界历史社会给了国家太多限制,国家给了政府和领导人太多限
制……而凡人平民无疑是最为受限和苦难的。在这些丰富的张力之下,若有一只上帝之
眼去观看任何一人的一生,那都是深刻丰富和辉煌的悲喜剧呢!

% 不过任何层面总有点光明可以逃逸出重重限制,总有希望,总有多样可能。

人类世界一直缺乏直面社会现实的勇气,常有一层幻梦面纱蒙在真实和悲苦之上,这层
面纱或遮掩或美化或扭曲……中国在数千年沧桑历史中形成的一种绝望使太多中国人尤
爱幻梦,常常自愿抛弃或无视现实悲剧的深刻、丰富和辉煌。

李大钊先生早在百年前就说:
\begin{quotation}
  \textbf{中国自古昔圣哲,即习为托古之说,以自矜重}:孔孟之徒,言必称尧舜;老
  庄之徒,言必称黄帝;墨翟之徒,言必称大禹;许行\footnote{先秦农家学派代表人物}之徒,
  言必称神农。此风既倡,后世逸民高歌,诗人梦想,大抵慨念黄、农、虞、夏、无怀、
  葛天的黄金时代,以重寄其怀古的幽情,\textbf{而退落的历史观,遂以隐中于人心。其或
    征诛誓诰,则称帝命;衰乱行吟,则呼昊天;生逢不辰,遭时多故,则思王者,思
    英雄。}而王者英雄之拯世救民,又皆为应运而生、天亶天纵的聪明圣智,而中国哲
  学家的历史观,遂全为\textbf{循环的、神权的、伟人的历史观}所结晶。一部整个的中国史,
  迄兹以前,遂全为是等史观所支配,以潜入于人心,深固而不可扰除。
\end{quotation}
今日又改变了多少呢?

很多人其实不是不知道,不是不能知道,只是不愿面对,做着迷梦。迷梦可以让人在世
界残酷时感受到并不存在的温暖。过去我认为这是虚弱,现在我可能更接受这是理性,
是智慧。面对了又能怎样呢?太阳照常从东方升起,一切又有何用呢?哪有什么迷雾外
的彼岸伊甸园?六朝何事,只成门户私计……

但我还是想在还能写的时候,把历史写下来。民为重,社稷为轻、君次之。是的,我想
写的是民的历史,而非人斗人或某些具体个人的历史。不管是社稷还是权力者都应当成
为民的历史的注解,而非相反。

或许一切都是徒劳,每朝每代都有人在旷野中呼唤;或许一切是最好,人性的缺点保证
了人类整体的繁衍生息,传续基因;或许这是不识庐山真面目的愚者所作所为……但我
还是想写这无用之书,总有人会去做这样的事。

本篇着重对事不对人,还请理智探讨。

% 很多国人常常呼唤一位
% 个人英雄,常常建立关于某个个人的偶像崇拜。在这种意识形态之下,我们常常使“自
% 己”缺场,,隐患颇多。

% 为应对悲苦人生,尼采所说日神阿波罗的幻梦便长期占据主要地位,现代性理性对其它理性
% 的压制又使之愈演愈烈。

% 只有敢于直面悲苦,尊重生命本身,尊重人之所以为人,在纵情忘我的对生命原始冲动、创
% 造的追求中才能去求得真实满足和慰藉。\footnote{这其中也暗含一种悲观的可能——人是否
%   能大量摆脱动物性和原始性……}


% 笔者最高学历只为初中,只是人间一介草民。按理说,笔者各方面素养完全不足以较好完成
% 对中国方方面面的论述。那么为什么还要完成本篇?

% 相较知识、政治、经济界人,我的草民身份和个人特质使我更为敢言、直接,更想寻求客观
% 公正描述,更为注重凡人在时代波涛中的跌宕起伏。

% 国人常将国家作为坚不可摧,或可凭高层领导的英明步步走高,这是不科学的。国家面
% 临的是一个错综复杂实践中的现实社会,有些选择必须要面对,但这些选择的选项在一
% 些时候可能都是灰色的,无一较为完善,如薄一波在《若干重大决策与事件的回顾(上
% 下)》形容统购统销时所说“两种`炸药'中的选择”。\pagescite[][259]{boyibo}

% 中国不能再激荡,中国不能乱!有问题和分歧要探讨和争辩,可以声音大些但要理智,
% 尤其要综合全局。人之艰难、国之不易,上层下层均要互相考量和体谅。拥护党和国家
% 的领导,采用合适渠道、方法是前提,过度激进要不得。


% 在这种民的历史之下,我们将可以得到一个更为清晰醒目的认识,再用这种认识去造福于民,
% 寻得社会的真正进步。我的所作,将有诸多错误和不足。我之所以自不量力去从事历史描述,
% 因为我相信我的作品最重要的是激发大众和他人的思考,激发他人对我所作的批判,从而实
% 现独立之个人,自由之思想,光明之未来。即使拙作无人问津,毫无影响力,也已在这条道
% 路上投下一颗石子。


% “关于中国”这部分着重于通过理清中国历史脉络以及笔者对当今中国问题的个人不成熟见
% 解来激发人的思考,从而减弱幻梦、尊重真实、勇敢面对、正视错误、科学发展、强盛国家、
% 保障人民以及为了人类的美好明天。这一目的并非有笔者亲自实现,实际上这系列文章应该
% 是几乎无人问津外加毫无影响力,但笔者可以成为立志于从事这方面的一系列推动者中的一
% 员。

% \section{本篇文章结构}

% 第x章到第x章为总结当代中国从1949建国年至今的经济重大事件,第x章为总结说明和笔者对
% 当今中国的思考建议。本系列着重参考了薄一波《若干重大决策与事件的回顾(上下)》,
% 邓书杰,李梅,吴晓莉和苏继红所著《中国历史大事详解丛书(套装共9册
% )》\cite{zhengshujie}\footnote{本套书为电子书,无法注明具体页码},blablablabla以
% 及对这些参考资料的追根溯源。



%%% Local Variables:
%%% mode: latex
%%% TeX-master: "../main"
%%% End:
