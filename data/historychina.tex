\chapter{计划经济时期 1949--1978年}

\section{社会主义过渡时期 1949--1956年}
\label{sec:hongguodu}

1949年新中国成立到1956年社会主义改造完成,是新民主主义社会向社会主义社会过渡的时期。

\begin{enumerate}
\item 1949年10月至1953年,过渡时期,逐步实现国家的社会主义工业化,并逐步实现国家对手工业和资本主义工商业的社会主义改造。

\item 1950年2月13日至15日,第一次全国财经会议,中央统管全国财政收支、国家国营贸易和物资调度,稳定金融物价,具体实施中实行了严格的公粮入库制度。薄一波认为这是新中国财经战线三大战役中的第一次大战役。

\item 1950年3月16日,第一次全国统战会议,人民民主统一战线。针对党内左倾倾向,毛泽东提出现阶段是工人阶级、农民阶级、小资产阶级和民族资产阶级四个阶级的联盟,应求得四个阶级的共同解放。

\item 1953年2月15日,中央正式决议,重点发展以土地入股为特点的农业生产合作社,左倾右倾均要不得,另外推广国营农场。1953年底到1955年春这一年半的时间里,合作社数量就由1.4万个发展到67万个。1956年1月,中央发布《1956--1957年全国农业发展纲要(草案)》,强调完成初级合作社到高级合作社的升级。1956年底,全国建成75.6万个农业生产合作社,高级社农户占农户总数的88.8\%。这标志着中国农村在生产资料所有制方面的社会主义改造基本完成。

\item 1952年3月,“五反”运动中,全国范围内形成了一个反对资本家“五毒”行为的斗争高潮。1952年10月,五反运动结束,中共中央转批的廖鲁言报告中指出
  \begin{quotation}
    根据华北、东北、华东、西北、中南5大区67个城市和西南全区的统计,参加“五反”运动的工商户总计999707户,受到刑事处分的只有1509人,仅占总数的0.15\%,其中判处死刑和死缓的仅19人,占判刑总数的1.26\%。(笔者注:这份报告未提受到非刑事处罚的人数、比例,以及对他们的惩罚措施。)
  \end{quotation}

\item 计划经济时代医疗制度是贯穿的,因此做下全面陈述,以下内容根据王晓玲所作论文《中国医疗市场政府管制的历史演进及制度反思》\cite{yiliaoshi}:

  1951年起开始对工商业职工及其供养的直系家属实行源自企业纯收入的\textbf{劳保医疗}。1952年底,全国90\%以上的地区建立了县级卫生机构,县级卫生院达到2123所。政府在1952年,1960年和1972年三次大幅降低医疗价格收费标准,低于成本部分进行财政补贴。1952年,开始实行针对国家工作人员的\textbf{公费医疗},后于1953年扩大至大学和专科院校。1955年在山西省高平县率先实行了医疗合作社和生产合作社相结合的\textbf{集体医疗}保健制度,标志着我国农村正式出现具有保险性质的\textbf{合作医疗}制度。1968年,毛泽东批示了湖北省长阳县乐园人民公社举办合作医疗的经验,\textbf{合作医疗}制度在全国蓬勃发展起来。计划经济时期禁止私人资本进入医院。


\item 第一个五年计划(1953-1957)出台,首要重点发展重工业。
  \begin{quotation} “一五”计划的主要指标是:五年内经济和文化教育建设总支出为766.4亿元……其中,基本建设的投资总额为427.4亿元,占55.8\%,其余44.2\%的资金计339亿元,用于基础建设所要的资源勘探、工程设计和器材储备等;工业交通运输和邮电业的设备大修、技术组织措施、新产品试制等;各经济部门的流通资金;经济和文化事业,培养专业干部等。

    基本建设投资427.4亿元的分配是:工业部门为248.5亿元,占58.2\%;农、林、水部门为32.6亿元,占7.6\%;运输邮电业为82.2亿元,占19.2\%;文教卫生部门为30.8亿元,占7.2\%。
  \end{quotation}

\item 1953年10月,国家对粮、油、棉统购统销。1955年3月3日,国务院发出《关于迅速布置粮食购销工作,安定农民生产情绪的紧急指示》,决定实行粮食“三定”(定产、定购、定销)。薄一波写到:
  \begin{quotation}
    继稳定物价、统一全国财经工作之后,被称为新中国财经战线上的第二次大战役的,就是1953年开始的对粮食等主要农产品实行统购统销(加上对资本主义工商业和个体农业、个体手工业的社会主义改造,就是财经战线的“三大战役”)。到1985年改行粮棉合同定购制度为止,这个在特定条件下开始实行的农产品统购制度,持续时间长达32年之久。而统销制度的一些基本内容现在还在持续进行。\pagescite[][255]{boyibo}
  \end{quotation}

  关于统购统销的研究存在多方论述,可参考田锡全论文《粮食统购统销制度研究的回顾与思考》这篇理论综述。笔者认为事实其实是比较清楚的,不去抠初始动因(可能是粮食购销形式严峻)及过分细节的话,统购统销在实际发展上仍然是利用及造成工农业剪刀差,在国家的强力计划管理之下发展重工业。正如薄一波所说,虽然党内外批评和反对苏联利用剪刀差的一些具体做法,但仍然“在实际上无法同剪刀差真正决裂”。\pagescite[][281]{boyibo}黄宗智提出“三农政策不仅把小农家庭农场经济纳入国家计划,实际上还强有力地把农民推向集体化的道路。”\pagescite[][175]{sanjiaozhou} ,国家依靠农村合作社等集体模式获得了对农村的前所未有的强力管理,另外在国家对合作社提供优惠政策及农民个人无法承担征购巨大压力的情况下,使农民主动融入合作社。

\item 1953年6月,中共中央起草《关于利用、限制、改造资本主义工商业的意见》。1954年9月2日,政务院颁布《公私合营工业企业暂行条例》。1955年10月全国工商联合会议通过了《告全国工商界书》。1956年初,全国范围出现社会主义改造高潮,资本主义工商业实现了全行业公私合营。

  % 1966年9月,当局按照原定的向资本家支付定息的年限已满[可疑 –讨论],决定不再支付定% 息,公私合营的企业就变成了完全社会主义性质的全民所有制企业。有报道称,按现时的% 概念,即一夜之间股民股票归公,房奴房产归公。未经任何合法手续,私营股份被“没% 收”为国有,公私合营企业全部变成了国营企业。[2]

  % 1979年1月,中央出台《党对民族资产阶级政策问题》规定:“公私合营时股票股息发 放% 到1966年9月结束,现有资产阶级工商业者要求领取在此前应领未领股息是可以 的”。但% 国家财政部又在当年下发文件,确定不再清退私股股金。[3]

  % 1983年2月,中共中央统一战线工作部和商业部联合发文规定:“国家已按年息五厘发 给% 定息,发至1966年3季度,公私合营资产(包括核定投资房屋)已属国家所有,不应退还% 本 人”。此后全国发生多例私股定息或股权的诉讼,皆因上述政策文件的原因而败诉。有% 学者对这一“不应退还”政策提出了质疑。既然向私股股东支付“定息”,就说明 “公% 家”承认私股股东对于合营财产的所有权。自1966年9月之后不再支付定息,并不说明一夜% 之间这些财产收归国有。[2]

\item 1956年是“一五”计划第四年,在这一年全国工农业总产值已经完成整个“一五”计划指标。工业方面总产值平均年增长19.2\%,超过目标14.7\%。在1957年“一五计划”完成时,农业方面:
  \begin{quotation}农业增长率为4.5\%,虽然完成了计划指标(4.3\%) ,但是与工业高速增长相比明显滞后。这也造成了农村购买力增长缓慢,原计划农村购买力增长100\%,实际上只增长20\%,大大低于计划目标。\cite{shiyiwu}
  \end{quotation}

\end{enumerate}

1956年的国民收入中,有92.9\%来自于公有制经济(国营、合作社、公私合营),这标志着中国已经从新民主主义社会过渡到社会主义社会。

\section{红色社会主义社会时期 1957--1978年}


笔者将1957--1978年称为红色社会主义社会时期。

\subsection{赫鲁晓夫的秘密报告}

对于新中国这一阶段历史的理解,必须加入对赫鲁晓夫秘密报告影响的认识。

1956年2月25日,赫鲁晓夫在苏共第二十次代表大会上作了《关于个人崇拜及其后果》的“秘密报告”,大力批判了斯大林,斯大林模式中一些残酷问题暴露出来,共产主义世界受到极大冲击,波兰发生波兹南事件,匈牙利发生十月事件,拥有绝大部份法国知识分子的法共有半数以上党员退党,其中一些理论家后来成为后现代主义的中坚力量。

毛泽东对斯大林的评价是功过三七分,功大于过。\footnote{此类文献较多,可简单参考\url{https://www.wxyjs.org.cn/mzdyj/201802/t20180208_236685.htm}。}笔者认为,毛泽东与斯大林之间存在分歧和矛盾,但这种分歧和矛盾是建立在两人同一个社会理想的框架之下,并且在当时“苏联的今天就是我们的明天”的语境之下,斯大林的被批判使毛泽东感觉到了沉重的危机,他阶级斗争的这根弦更加紧绷起来,整治党内外,建立党内外建设社会主义、迈向共产主义的统一共识成为毛泽东的重要目标。但是也要考虑在历史语境下,当时世界各国各派领导人的神经都高度敏感和紧张。

\subsection{整风到反右派}

据中共中央文献研究室编写的《毛泽东传(1949-1976)》,1956年下半年开始,
\begin{quotation}在半年内,全国各地,大大小小,大约有一万多工人罢工,一万多学生罢课。从1956年10月起,广东、河南、安徽、浙江、江西、山西、河北、辽宁等省,还发生了部分农民要求退社的情况。
\end{quotation}

1957年4月27日,中央起草了《中央关于整风运动的指示(初稿)》,决定在全党的范围
内“重新进行一次普遍的、深入的反官僚主义、反宗派主义、反主观主义的整风运
动”。5月4日,毛泽东起草《关于继续组织党外人士对党政所犯错误缺点展开批评的指示》,
邓小平、薄一波对于此时毛泽东的评价是正确的——毛泽东此时确实是将党内官僚、宗派、主
观问题当作社会波动的主因,想靠开放批评来进行党内的整治。党外人士的批评中开始出现
很小部分反共反社会主义思想,5月15日,毛写了《事情正在起变化》,认为一些右派正在猖
狂进攻。6月8日,毛为中央起草了《关于组织力量准备反击右派分子进攻的指示》,直言中
国如果不反右,中国可能面临匈牙利十月事件……到1958年8月为止,这一年半的时间里,本
是党外批评人士采用的“大鸣、大放”形式被我党和政府借用了过来,成为“大鸣、大放、
大争辩、大字报”的反对他们全体并且犯了严重扩大化错误的右派运动。据官方公布,这一
运动中全国实际划归右派分子55万多人,其中绝大多数人被错划。

薄一波认为,1956年9月党的第八次全国代表大会正确提出的“国内的主要矛盾,已经是人民
对于建立先进的工业国的要求同落后的农业国的现实之间的矛盾,已经是人民对于经济文化
迅速发展的需要同当前经济文化不能满足人民需要的状况之间的矛
盾”被1957年9月到10月9日召开的第八届中央委员会第三次会议上“无产阶级和资产阶级的
矛盾,社会主义道路和资本主义道路的矛盾,毫无疑问,这是我国社会的主要矛盾。”错误
取代。自社会主义改造完成后,更应依靠法制解决社会矛盾,而非群众性阶级斗争。从整风、
反右派开始到1976年“十年动乱”结束前,我党仍然错误采用了群众性阶级斗争的形
式……自1958年到1978年十一届三中全会的20年,党和国家错误实行了“以阶级斗争为
纲”的方针。\pagescite[][620-632]{boyibo}

% 十一届六中全会的决议指出:“在社会主义改造基本完成以后,我国所要解决的主要矛盾,% 是人民日益增长的物质文化需要同落后的社会生产之间的矛盾”。又说:“在剥削阶级作% 为阶级消灭以后,阶级斗争已经不是主要矛盾。由于国内的因素和国际的影响,阶级斗争% 还将在一定范围内长期存在,在某种条件下还有可能激化。”

\subsection{浮夸风、命令风、共产风、生产瞎指挥风、干部特权风}

\begin{enumerate}
\item 在经历了一年半左右的对“反冒进”的批评和提出大跃进之后,1958年5月5日至23日召开的中共八大二次会议上提出“鼓足干劲、力争上游、多快好省地建设社会主义”的\textbf{总路线},在钢铁等重要工业品的产量上赶超英美成为一个目标,笔者认为这标志着在中央政府层面上正式开展大跃进运动。虚报目标并次次层层地加码,“浮夸风”盛行。薄一波认为农业上的“浮夸风”形成的盲目乐观又导致了国家经济重心转向工业,成为了工业上的“浮夸风”,导致举国上下“以钢为纲”的大炼钢铁,农村砸锅进行土法小高炉炼钢,钢企不重安全和质量地快速出钢。

  \begin{quotation} 1958年8月中共中央通过了国家计委重新拟定的《关于第二个五年计划的意见》(1958--1962年的国家计划,这也是真正开始实施的计划),提出了天方夜谭的高指标,冒进指数(原文注释:冒进指数是指本计划期的指标值相当于上一个计划期实际值的百分比,该比值越高,说明制定的计划越冒进,反之,越保守。)达到354.6\%,基本建设投资规模是“一五”时期的7.8倍,工业总产值增长速度是4.9倍,农业总产值增长速度是6.7倍。\cite{shiyiwu}

    由于“大跃进”浮夸风的影响,1959年全国定产指标为5000亿斤原粮,而1959、1960、1961年的实产量分别只为3400亿斤、2870亿斤、2950亿斤。三年平均实产比1957年减少827.6亿斤,但平均年征粮食却比1957年增加了95.8亿斤,相当多的地方购了农民的“过头粮”。\footnote{这一数据得到了1991年版胡绳所著《中国共产党的七十年》的支持。}\pagescite[][278]{boyibo}
  \end{quotation}


\item 浮夸风的盛行,导致了中央的盲目乐观,认为“共产主义在我国的实现已经不是什么遥远将来的事情了”,,又刮起了“共产风”。浮夸风导致的现实具体因素方面,薄一波书中认为是农田水利的大规模建设要求有更为行之有效的基层组织结构管理,对基层组织结构规模提出了较大要求。1958年3月20日成都会议通过,同年4月8日政治局会议批准的《中共中央关于把小型的农业合作社适当地合并为大社的意见》佐证了这点。\pagescite[][728-730]{boyibo} 1958年8月17日到30日,中央政治局扩大会议通过《关于在农村建立人民公社问题的决议》。9月1日《红旗》杂志第七期中的嵖岈山卫星农业社模式推广至全国。生产生活资料公有,公社命令式调拨农民人力、物力、财力,吃饭不要钱\footnote{薄一波写道,国家统计局1960年1月报告,参加公共食堂吃饭的约4亿人,占人民公社总人数的72.6\%。},个别公社收缴了农户土地、房屋、资金、粮食……邓书杰等作者在书中写到
  \begin{quotation}到1958年10月底,全国农村就实现了人民公社化。全国原有的74万多个农业生产合作社,此时改组成了2.6万多个人民公社,加入公社的农户达1.2亿,占总农户数的99\%以上。
  \end{quotation}

  农村人民公社兴起的同时,城市人民公社也开始兴起。据邓书杰书中所说
  \begin{quotation}到1960年7月底,在全国190个大中城市中,已经建立了1064个人民公社,基本上实现了城市人民公社化。
  \end{quotation}

  人民公社所出现问题的宏观根源究竟是什么?其实早在1958年11月2日至10日召开的第一次郑州会议上已经由毛泽东本人提出答案,并在1958年11月28日至12月10日的中共八届六中全会得到深化。八届六中全会作出《关于人民公社若干问题的决议》,提出了问题的根源,笔者认为时至今日这仍然高度接近标准答案。

  笔者在苏联一章阐述过马克思的科学社会主义的目的论和空想成分,马克思的历史唯物主
  义足以对他的共产主义作出批判,在此不再赘述。除此之外,这一时期的冒进无视了历史
  唯物主义中所要求的生产力和生产关系等物质基础,混淆了集体所有制和全民所有制,混
  淆了社会主义初级阶段和共产主义社会。人民公社的实质仍是集体所有制,且是可以无限
  统领全民的集体所有制。集体所有制的权力主体不是全民,而是集体组织,这一环境之下
  政社合一的集体组织是公社及从而可调拨公社财产的县以上国家机构,官僚具有的强大权
  力和自身欲望是人民公社问题一个需要探讨的重要问题。

  另一方面,马克思的一些文章中已经作出一些判断,特别是在《哥达纲领批判》这一科学
  社会主义重要纲领文件中,沉重批判了拉萨尔为首的德国社会主义工人党。《纲领中》明
  确提出共产主义第一阶段(即社会主义阶段)仍是按劳分配和“带有经济、道德、精神方
  面的资本主义痕迹”,“仍带有资产阶级权利”,“权利决不能超出社会的经济结构以及
  由经济结构制约的社会的文化发展。”。\pagescite[][435]{maenwen3}当时苏联历代领导
  人均犯过冒进的错误,很遗憾中国未能从苏联历史经验中吸取足够教训。

  更为遗憾的是党中央未能发展和坚持《决议》的正确看法。1959年7月2日至8月1日,中共
  中央政治局在江西庐山召开扩大会议。会议初期朱德、周小舟、周慧等对公共食堂提出一
  些反对意见。7月14日,因彭德怀写信给毛泽东,更为激进的提出左倾错误。7月21日,张
  闻天会议上讲话肯定和支持彭德怀观点。7月23日,毛泽东发表长篇讲话,批评彭德怀。
  自此,本次大会从“纠左”又转向了“反右”,左倾错误继续扩大化。

  另外笔者想提一点,据《重整河山1950-1959》与《动荡年代960-1969》一书,
  \begin{quotation}(农村人民公社)大办公共食堂、幼儿园、托儿所、幸福院等公共事业。截至1958年末,全国农村共建立公共食堂340多万个,托儿组织340多万个,幸福院15万所。

    (城市人民公社)在这些基本实现了城市人民公社化的大中城市中,共计有850万闲散劳
    动力被安置就业,占这些城市闲散劳动力总人数的87\%;共计兴办了7.6万个居民公共食
    堂,就餐人数达1700万;8.8万个托儿所,入托儿童为365万;还建立了8.9万个服务站。
  \end{quotation}
  虽然这些“大跃进”充满了与生产力不足的矛盾和大规模调用各方民众资源的问题,但由
  此可见毛泽东的政治理想和抱负。


\item 1959--1961年\footnote{一说是1958--1962年大饥荒。},三年困难时期,岁大饥,人相食,饿死者以千万计。


\end{enumerate}

\subsection{大跃进中后期的安徽省}

之所以选择安徽作为单独一小节,原因有四:一,机缘巧合。二、虽就饿死人数存在各方争议,但对死亡人数和比例的省排行来说应基本无差别。综合各方观点,安徽省总饿死人数不及四川、位居全国第二,但是饿死人比例位居各省最高。三、安徽在大跃进前后出过两个上达天听的大人物,一个张恺帆,一个曾希圣,两人均具各方面代表性。四、安徽省官方文献较为健全。

曾希圣于1952年1月--1962年2月任职安徽省委第一任书记,1960年10月,兼任中共中央华东
局第二书记、中共山东省委第一书记,\footnote{1961年1月曾希圣主动申请辞去山东省委第
  一书记职务。普遍说法是曾希圣希望大搞“责任田”,无心兼任两省第一书记。另一说法
  是中央监察委员会针对安徽省饿死人进行的调查促使曾希圣辞职,这一说法可见尹曙生所
  作《曾希圣是如何掩盖严重灾荒的》\url{http://www.yhcqw.com/33/10039_2.html}。另
  外曾希圣1962年以后的履历表不健全,请读者帮忙提供。}也曾任济南军区政治委员等职。
大跃进时期曾希圣在钢铁、水利和粮食方面均跟随“浮夸风”。
\begin{quotation} 1958年安徽产粮167.9亿斤,却被浮夸虚报成450亿斤(指标更高,是494亿斤),谎报了2.68倍。\cite{zhangfandang}

  在上下不讲真话的氛围中,1959年安徽粮食生产任务,于3月30日向全省宣布:“超额完成720亿斤”!这一严重浮夸的高指标,是1959年实际产量(140.2亿斤)的5.14倍!虽然当年征购粮是70.94亿斤还不到指标的9.9\%,但却占实际产量的50.6\%。为此,安徽人民蒙受了巨大的痛苦,出现了严重的饿、病、逃、荒、死。\cite{zhang1959}
\end{quotation}

张恺帆1959年时任安徽省三把手——安徽省委书记处书记,安徽省副省长,在基层考察饿死人情况后,于1959年7月未经组织程序自作主张停办无为县6000多公共食堂,开仓放救济粮\footnote{笔者曾在某文档看到,某地被划拨救济粮5000余斤,实收1500斤左右,这份资料如有读者知晓,还望告知。},无为县问题也得到了时任安徽省书记处候补书记、副省长陆学斌的支持。在8月、9月召开的两次安徽省委会议上两人被定性为“张凯帆、陆学斌反党联盟”。
\begin{quotation}后来平反时统计:仅无为一县,因张恺帆事件受株连被批斗、被处理的县、社、队党员、干部和群众,共达28741人。\cite{zhang1959}
\end{quotation}

因笔者资料有限,不知酷爱仕途的曾希圣为何突然在1960年底开始关注并顶着重压积极实施“责任田”,笔者愿意相信曾希圣此时已无法再承担良心的拷问。

老农刘庆兰父子1956年起先后上山独立垦荒,4年“共向集体无偿缴纳上交粮食4716斤(平均每年1179斤)”。\cite{anhuiliushi}\footnote{关于刘庆兰事迹,邓书杰的《中国历史大事详解丛书》与江鲲池《60年代初曾希圣在安徽推行责任田始末》说明相同,并与时任安徽省委农工部工作人员陈者香的会议,与陈大斌对刘庆兰儿子刘志立的采访论述有出入,但并无根本性的颠倒,应是邓书杰和江鲲池的无主观故意的数据采用错误。}刘庆兰的事迹引发曾希圣极大兴趣,后经毛泽东“同意试验”,在湖北进行“计划统一、分配统一、大农活和技术性农活统一、用水和管水统一、抗灾统一”等“五个统一”之下的“责任田”推广,可以认为这“五个统一”是支持“责任田”官员心目中的“社会主义屏障”。

\begin{quotation}它一问世就很受农民欢迎,全国不少地方都程度不同地实行起来。比如,当时搞各种形式包产到户的,安徽全省达80\%,甘肃临夏地区达74\%,浙江新昌县、四川江北县达到达70\%,广西龙胜县达42.3\%,福建连城县达42\%,贵州全省达40\%,广东、湖南、河北和东北三省也都出现了这种形式。据估计,当时全国实行包产到户的约占20\%。\pagescite[][1078]{boyibo}
\end{quotation}

曾希圣在七千人大会上受到批判,较多人,包括薄一波和陈者香等人认为曾希圣是因这种“单干风”而被调离安徽,笔者持不同意见。当时是1962年1月30日七千人大会继续开“出气会”,分派刘少奇、周恩来、邓小平、朱德、陈云等再次去几个省区大组,组织开展县委和地委对省委书记的“出气会”,且主要矛头均集中在“大跃进”的错误。各省委书记捶胸顿足者有,嚎啕大哭者有,安徽省委第一书记曾希圣被批判的主要错误是他的饿死人掩盖子问题。如果将其说成因“单干风”而被批判,即使结合安徽省1962年3月份开始取消“责任田”,时间点上与邓子恢、陈云等人在七千人大会同样主张搞“单干风”,毛泽东后派田家英进行“责任田”调查也有时间上的冲突,笔者就此认为曾希圣主要因为“责任田”问题受到批判难以服人。

\begin{quotation}在1962年七千人大会上,刘少奇同志参加安徽组讨论,追问安徽饿死了多少人,第一次报40万,后来追问紧了,报到400万。实际上约有500万人。(此话据网载,出自《张恺帆回忆录》,第344页,安徽人民出版社,2004年10月第一版。)
\end{quotation}

张恺帆的500万数据可由以下数据佐证,据《安徽省志·人口志》所载表1--1--14,安徽省总人口在1959--1961年间负增长406万,仅1960年相比1959年就负增长11.21\%,减少3839979人,但这个人数减少包括人口外流。\pagescite[][27]{anhuishengzhi} 《安徽省志·人口志》表2--1--18记载1960年实际死亡人数据为2218280人,死亡率为68.58‰,书中还就此写到“此为年报统计数,人口实际损失更大”,1960年的死亡比例远高于建国后第二高峰——1959年的16.72‰。从1961年至1985年,安徽省年死亡人数再也未超过300000人,死亡比例最高为1964年的8.6‰。\pagescite[][95-96]{anhuishengzhi} 笔者认为同样要引起注意的是,据《安徽省志·人口志》,在大跃进拨乱反正3年后的1964年仍出现了1.81\%的负增长。本书解释是1964年全国第二次人口普查时,各地虚报人口获取利益的现象被纠正,如果将1964年普查到的负增长结合1960年的负增长,其中蕴含的其它可能令人深思。1958年数据显示溺婴死亡10159人,1959-1961年溺婴数据未提供……\pagescite[][108]{anhuishengzhi}根据周曼硕士论文《三年困难时期安徽人口变动研究》,“关于安徽省的婴儿死亡率,也没有找到确切的数据。安徽与河南同属重灾区,再根据两省死亡率的比较,安徽省的婴儿死亡率可能会高于河南省(河南省1960年婴儿死亡率为276.8‰),但确切数据无法估计。”


\begin{quotation}
  1960年,人口死亡异常,死亡率在10\%以上的有太和县(163.47‰)、无为县(158.29‰)、宣城县(147.26‰)、毫县(145.95‰)、宿县(130.32‰)、凤阳县(119.46‰)、阜阳县(118.31‰)、肥东县(113.31‰)、五河县(108.71‰)等9个县,死亡率最低的是合肥市(11.27‰)。\pagescite[][98]{anhuishengzhi}
\end{quotation}

据姚宏志论文《1959--1961年安徽灾荒的差异性分析》,安徽省大饥荒时期烤烟、棉花等经济作物国家统购价格下降20\%,改种粮食较多。文中引述的韩敏、王朔柏和陈意新等人的调查结果显示,宗族关系中官僚多的村庄受灾影响小,外派官僚取代原生宗族领袖的村庄受灾影响大,原生宗族领袖和宗族关系未变的村庄受灾影响小(偷稻种、瞒产、藏粮、对外守口如瓶等)。另外,
\begin{quotation}(安徽省)各市县地方志关于人口死亡率的数据普遍(比《人口志》)更高。以宣城县为例,安徽省《人口志》显示,1960年全县人口的出生率为4.57‰,死亡率为163.47‰,两相比较,人口自然增长率为-142.69‰。《宣城县志》没有显示当年全县人口出生率和死亡率数据,但提供了人口自然增长率数据,为-205.88‰。\cite{zaihuangchayixing}
\end{quotation}

曾希圣在上下层均力排众议推行“责任田”有功,但又何必将其粉饰为神呢,“出气会”初期依然死扛强硬,他能算是个神吗?即使经过了中国历史几千年的教训,国民仍常常打破一个偶像后,又树立起一个新的偶像来。什么时候,我们才能学会实事求、客观求真与就事论事呢?其实就算是张恺帆,笔者也觉要敬佩他停办食堂等的事迹和操守,也莫要让其成为神。我们要看的应当是人所做之事,而非赋予人一个标签……对于神圣者和偶像的崇拜,无外乎是将本人无力之事寄托于他人,自己的人格在此情境下无论如何也是残缺的。

这次始于1月30日的“出气会”,高层领导的参与除具备打破僵局、保证对省委书记批判展开的作用外还有一个作用,就是防止批判扩大至对一些省委书记更为严重的追究责任,如入狱、枪毙。不管对曾希圣等省委书记的“出气会”是毛还是刘的主导,不管这是否阴谋论中的一部分,它这一环本身的作用笔者个人认为是清晰的,那就是将中央责任大部份转移到省委一级。

另外省委书记被地委、县委干部弄得焦头烂额,可这地委、县委干部就是清白被迫的吗?

最后,笔者也向曾希圣道一声歉意。本篇文章前后虽批评很多人,但只细述了他一人的过失。在时代的大背景下,曾希圣也只是个浮萍。笔者细述他的深层目的不是批评他个人,而是他前后的故事均较有代表性。曾希圣,来生还做曾希圣吗?

\subsection{对“大跃进”的补救措施及反思}

\begin{enumerate}

\item 李若建论文《权力与人性:大跃进时期公共食堂研究》将此时国人分为六个阶层:高级、中级、基层官员、利益相关民众和利益不相关民众,分别就权力和人性两方面展开论述。笔者认为这是一篇不可多得和很客观的论文,建议大家阅读。李若建另一篇《理性与良知:“大跃进”时期的县级官员》则说明了县官的生态环境、正职和副职差别和一些县长的英雄事迹,前文于无意中也引用了李若建《困难时期的精简职工与下放城镇居民》。我们还有一些负责任的官员、专家、学者和人民背负着国计民生的伟大理想和目标,向他们致敬。因笔者知识面狭窄,无法将他们姓名一一道出,还请读者自我发掘。

\item 1960年11月3日,中共中央向各级党组织发出《关于农村人民公社当前政策问题的紧急指示信》(简称十二条)和《中共中央关于贯彻执行“紧急指示信”的指示》,反五风,清理“一平二调”,反对五风,明确“以生产队为基础的三级所有制(公社、生产大队、生产队),是现阶段人民公社的根本制度”,城乡精简,“允许社员经营少量的自留地和小规模的家庭副业”等。庐山会议后被中断的“纠左”重新起步,此后一系列会议基调在此层面上有所扩展,笔者不再赘述。郑文中认为,“确切意义上的调整即后退(全年基建、钢产量、粮食产量指标下调),是从1961年9月庐山工作会议开始的。”经过“五风大跃进”,国家吸取了蔑视客观生产力的教训,即使在“文化大革命”时仍要“促生产”,生产力、特别是农业虽再发生过动荡,但再无这样强度的惨剧发生。


\item 城乡二元结构进一步隔离。
  \begin{quotation} 1956年秋天,由于过激的合作化运动加上自然灾害,导致不少省份粮食歉收、农民吃饭成问题,安徽、河南等省出现大量农民外流,进城寻求就业机会。在这种情况下,12月《国务院关于防止农村人口盲目外流的指示》出台,防止农村农民进城就业。

    (出台一系列此类政策之后,)1958年1月9日,全国人民代表大会常务委员会第九十一次会议通过并颁布了新中国第一部户籍制度《中华人民共和国户口登记条例》……正式确立了户口迁移审批制度和凭证落户制度。以这个条例为标志,中国政府开始对人口自由流动实行严格限制和政府管制……第一次明确将城乡居民区分为“农业户口”和“非农业户口”两种不同户籍。严格限制农村农民迁往城镇,限制城市间人口流动,在城市与农村之间构筑了一道高墙,城乡分离的“二元经济模式”因此而生成。

     (户籍制度变化)第二阶段:(1958年-1978年),这一时期包括大跃进、三年困难时期和十年“文革”。严格限制户口迁移特别是严格限制农民向城市迁移时期,严格控制农村人口流人城市,压缩城市人口,包括精简职工、知识青年上山下乡、干部下放农村等,出现了人类历史上罕见的人口从城市迁往农村的反向运动,形成了一整套严格的户籍管理制度。\cite{quxiaohuji}
  \end{quotation}

  精简职工方面,
  \begin{quotation}(大跃进时期的大招工)使得工人数从1957年的3101万增加到1960年的5969万,增长92.5\%。职工人数的增加,特别是从农村招收的职工,给城镇带来了大批的人口,1957-1960年间,中国的城镇人口从9949万增加到13073万,其中由农村迁入城镇的大约2218万。

    当粮食危机越来越严重时候,许多城市已经面临几乎没有库存的窘境,1960年底全国82
    个大中城市的库存粮食只有正常水平的 $1/3$。1960年6月北京、天津和辽宁的几个主要城市的库存粮食几乎没有,只能维持不到10天的供应,上海的大米库存已经没有,天天告急。

    有关的统计,在1961-1963年间,压缩下放2500万城镇人口,精减职工1833万人,被精减的职工中,大部分也被下放到农村,少数转为城镇集体企业工人,还有少数流浪到边疆地区,在当地谋生。\cite{jingjianzhigong}
  \end{quotation}

\item 1962年1月11日至2月7日,扩大的中央工作会议在北京举行,俗称“七千人大会”。七千人大会是一次纠左的、带有很多正面影响的会议,同时也是一场涉及政治、经济、权力纠葛的复杂会议。

  刘少奇代表中央引一老农的话“三分天灾,七分人祸”,也引用毛泽东说过几次的“指头论”对左的做法进行了批评,毛泽东带头进行了自我批评,众高层官员也开展批评和自我批评。

  1962年2月21日至23日,刘少奇主持召开中共中央政治局常委扩大会议(西楼会议)。深化和发展了七千人大会的“民主集中制”之风。七千人大会和西楼会议奠定了此后半年的进步纠左基调。

  邓子恢会后继续着力于推广“责任田”和“包产到户”。7月作出《关于农业问题的报告》。

  1962年2月27日,王稼祥、刘宁一、伍修权联名给中央写信,正确分析了资本主义和社会主义阵营间、社会主义主义阵营内部的缓和,提倡对国外援助要量力而行。

  “1962年4月27日,中共中央根据扩大的中央工作会议的精神,发出《关于加速进行党员、干部甄别工作的通知》”,邓小平主持了这次会议,并推动平反工作展开,”到1962年8月,全国已有600万干部和党员得到了平反”。

  1962年6月,彭德怀上书“八万言书”。

\end{enumerate}

\subsection{继续阶级斗争}

虽然此后一直坚持阶级斗争,但再也不敢像“五风”盛行时那样破坏生产了。

\begin{enumerate}
\item 七千人大会的胜利成果只维持很短时间,1962年8月在北戴河召开的中共中央会议
  和9月召开的中共八界十中全会上,重提“无产阶级和资产阶级之间的阶级斗争、社会主义
  道路和资本主义这两条道路的斗争。”

陈云、邓子恢和田家英调查后支持的“包产到户”、“责任田”被批判为“单干风”,刘少奇、周恩来、陈云等对经济困难形势的判断等被批判为“黑暗风”,王稼祥等被批判为“三和一少”,对右派分子的甄别平凡、彭德怀及受《刘志丹》一书牵连的习仲勋、贾拓夫、刘景范被定为“翻案风”。

\item 此后的历史,笔者认为已经乏善可陈。1963--1966年指向基层的“四清”,即城乡社
  会主义教育运动。1966--1976年的“文化大革命”。阶级斗争,一抓就灵。发动群众斗群
  众、官僚,发动官僚斗官僚。这时很难说其实质还是为了阶级斗争,而更像是家天下了。
\end{enumerate}

\subsection{笔者个人的总结}

\begin{enumerate}
\item 问题的深层根源,任何了解而不是号称了解马克思历史唯物主义的人均知道,历史唯
  物主义语境中,生产力和生产关系对于社会、国家和激进党派等上层建筑的限制都极为苛
  刻,后者几乎不能超越前者的制约。不管是马克思的科学社会主义理论部分还是中苏的科
  社实践部分,都有违背马克思历史唯物主义的限制要求之处。(对马克思的批判可参见笔
  者所写\namecref{sec:marxkexue}一节,苏联教训可参照\namecref{sec:beili}一节。)
  中国1957年后一系列运动的扩大化部分均错误想当然地实行了阶级斗争路线。阶级是现象,
  历史唯物是理论和实质上的探讨,共产主义带有空想成分,只提阶级和群众将彻底沦为唯
  心主义,而非唯物。

\item 空怀理想无论如何也接不了地气,它也不能成为善的代名词,实事求是是行动主义的
  必要前提、目标和标准。为了遥远伦理道德理想(空想社会主义)无视当前阶段的伦理道
  德——甚至肆无忌惮剥夺生命,这本身就是一个悖论。如钱理群所说
  \begin{quotation}
    至善至美的理想社会,只能存在于彼岸世界,是一个可望不可及的远方目标。它的存在,
    使人们永远也不会满足于现状,从而加强对现实的批判和改造力量。但这种改造、变革
    的努力,只能使此岸世界不断改善,不断趋向、接近理想,却永远也不会达到至善至美
    的理想境界。任何将彼岸世界此岸化的努力,都会给人类带来灾难,\textbf{天堂的现
      实化就是地狱}。
  \end{quotation}

\item “民主集中制”是显明的“民主的集中制”,集中制为主。笔者认可这一观点,中国
  必须实行集中制,不然国破家亡,各省市分裂,广大国民将承受更为惨重的灾难。在全球
  资本主义自由化大行其道的今日,大都市、大城市具有超越国家与全球化接洽并融合的欲
  望和能力,这一环境更使一个不集中的中国难以避免遭受多地多极分化、中央调度能力严
  削从而分裂的噩运。

  党和国家的高层领导人的回忆录中常提法制和民主成份的加入(“法制”今日应当变更
  为“法治”),但是问题在于,集中制为主下,民主与法治的成份到底该有个怎样的度,
  怎样来实现这个度?历史唯物主义是极为宏观的理论,它缺乏对这一层面的考量,实质上
  历史唯物主义本身带有对上层建筑的一些轻视。对于中国来说,官僚考核制度长远以来一
  直是一重大问题,它常常是唯上而不唯下、爱官而不爱民,别的国家也同样面临这一问题。
  我们该怎样发展官僚考核制度,是党和国家的重大课题。

\item 雪崩之下,没有一片雪花是无辜的。这段时期中,雪崩是主因、是制度,雪花是次次
  因、是人性。这段动荡年代的人性体现足以给我们太多的警醒和教训,一些人是那样的残
  忍、残酷、蔑视生命、伦理、公理。

  联合国多年来一直倡导个人主观能动性,便是想要个人突破各种现代理性限制,最终使进
  步的个人们彼此聚合为有力的、进步的和超越的成份。我们应当发展个人主观能动性,保
  持对人性恶的一面的警惕并敢于去对抗恶,有时甚至要为此抱有一种悲剧意识。本
  书\cref{chap:gerenshehuixue}也是基于此点。
\end{enumerate}

下一张我们来探讨中国市场经济体制过渡时期1978--1992年。
% http://econ.cssn.cn/jjx/xk/jjx_yyjjx/jjx_slyjsjjx/201310/t20131024_516814.shtml % 中国财政支出结构的演进研究(上) 穷富地区支出差异,行政管理比重的曲线。

% 下图反映的是中国20世纪50年代至80年代城镇人口数的变化情况。其中,导致从C到D变化% 的主要原因是 A.人民公社化运动B.“大跃进”的影响C.国民经济的调整D.自然灾害% % 的影响





%%% Local Variables:
%%% mode: latex
%%% TeX-master: "../main"
%%% End:
